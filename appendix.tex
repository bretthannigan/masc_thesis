%% The following is a directive for TeXShop to indicate the main file
%%!TEX root = diss.tex

\chapter{Derivation of Matrix Inequalities with One Quadratic Term}
\label{sec:apd-a}

\section{Derivation of \glsentryshort{GKYP} Inequality with Arbitrary $\mathcal{D}$}
\label{sec:apd-a-1}

\begin{thm}
	Equation~\ref{eq:lmiinf} from \autoref{sec:opt-gkyp} is equivalent to the following:
	
	\begin{equation} \label{eq:lmiinf-equiv}
		\begin{bmatrix}
			-\Xi_{11} + aa^T & -\Xi_{12} + a & -\mathcal{C}_q^T - a\mathcal{D}_{qp}^T \\
			-\Xi_{12}^T + a^T & -\Xi_{22} + 1 & -\mathcal{D}_{qp}^T \\
			-\mathcal{C}_q - a^T\mathcal{D}_{qp} & -\mathcal{D}_{qp} & \gamma_\infty
		\end{bmatrix} \geq 0
	\end{equation}
	
	where (\ref{eq:lmiinf-equiv}) contains just one nonlinear term in variable $a$, and:
	
	\begin{equation*}
		\begin{bmatrix}
			\Xi_{11} & \Xi_{12} \\
			\Xi_{12}^T & \Xi_{22}
		\end{bmatrix} =
		\begin{bmatrix}
			I & a \\
			0 & 1
		\end{bmatrix}
		\begin{bmatrix}
			\mathcal{A} & \mathcal{B}_p \\
			I & 0
		\end{bmatrix}^T
		\left(\Phi \oplus P_\gamma + \Psi \oplus Q_\gamma\right)
		\begin{bmatrix}
			\mathcal{A} & \mathcal{B}_p \\
			I & 0
		\end{bmatrix} 
		\begin{bmatrix}
			I & a \\
			0 & 1
		\end{bmatrix}^T
	\end{equation*}
	\begin{align} \label{eq:pq}
		P_\gamma = \gamma_\infty^{-1}P && Q_\gamma = \gamma_\infty^{-1}Q.
	\end{align}
\end{thm}

\begin{proof}
	Starting from \autoref{eq:lmiinf}, the procedure mentioned in \autoref{sec:opt-ss} is followed to eliminate non-convex products in the first term of the \gls{LMI} \cite[Th. 1]{Li2014}:

	\begin{equation} \label{eq:a-1}
		-\begin{bmatrix}
			I & a \\
			0 & 1
		\end{bmatrix}
		\begin{bmatrix}
			\mathcal{A} & \mathcal{B}_p \\
			I & 0
		\end{bmatrix}^T
		f\left(\Phi, \Psi, P, Q\right)
		\begin{bmatrix}
			\mathcal{A} & \mathcal{B}_p \\
			I & 0
		\end{bmatrix} 
		\begin{bmatrix}
			I & a \\
			0 & 1
		\end{bmatrix}^T +
		\ldots 
%		-\begin{bmatrix}
%			I & \vec{a}_S \\
%			0 & 1
%		\end{bmatrix}
%		\begin{bmatrix}
%			\mathcal{C}_y & \mathcal{D}_{yx} \\
%			0 & I
%		\end{bmatrix}^T
%		\begin{bmatrix}
%			1 & 0 \\
%			0 & -\gamma_\infty
%		\end{bmatrix}
%		\begin{bmatrix}
%			\mathcal{C}_y & \mathcal{D}_{yx} \\
%			0 & I
%		\end{bmatrix}
%		\begin{bmatrix} 
%			I & \vec{a}_S \\
%			0 & 1
%		\end{bmatrix}^T 
		\geq 0.
	\end{equation}

	Let the notation $\Xi_{ij}$ be used for the linear part:
	
	\begin{equation} \label{eq:a-2}
		-\begin{bmatrix}
			\Xi_{11} & \Xi_{12} \\
			\Xi_{12}^T & \Xi_{22}
		\end{bmatrix} + \ldots
%		\\
%		-\begin{bmatrix}
%			I & \vec{a}_S \\
%			0 & 1
%		\end{bmatrix}
%		\begin{bmatrix}
%			\mathcal{C}_y & \mathcal{D}_{yx} \\
%			0 & I
%		\end{bmatrix}^T
%		\begin{bmatrix}
%			1 & 0 \\
%			0 & -\gamma_\infty
%		\end{bmatrix}
%		\begin{bmatrix}
%			\mathcal{C}_y & \mathcal{D}_{yx} \\
%			0 & I
%		\end{bmatrix}
%		\begin{bmatrix} 
%			I & \vec{a}_S \\
%			0 & 1
%		\end{bmatrix}^T
		\geq 0.
	\end{equation}

	\autoref{eq:a-2} may undergo a congruent transformation by $\gamma_\infty^{-\frac{1}{2}}I$ introducing a commutable factor of $\gamma_\infty^{-1}$ to every element. For the first summation term, the factor is absorbed into $Q$ and $P$ with the redefinition from \autoref{eq:pq} yielding:

	\begin{equation} \label{eq:a-3}
		-\begin{bmatrix}
			\Xi_{11} & \Xi_{12} \\
			\Xi_{12}^T & \Xi_{22}
		\end{bmatrix} - 
		\begin{bmatrix}
			I & a \\
			0 & 1
		\end{bmatrix}
		\begin{bmatrix}
			\mathcal{C}_q & \mathcal{D}_{qp} \\
			0 & I
		\end{bmatrix}^T
		\begin{bmatrix}
			\gamma_\infty^{-1} & 0 \\
			0 & -1
		\end{bmatrix}
		\begin{bmatrix}
			\mathcal{C}_y & \mathcal{D}_{qp} \\
			0 & I
		\end{bmatrix}
		\begin{bmatrix} 
			I & a \\
			0 & 1
		\end{bmatrix}^T
		\geq 0.
	\end{equation}

	Multiplying the inner factors in the second term of \autoref{eq:a-3} leads to:
	
	\begin{equation*}
		-\begin{bmatrix}
			\Xi_{11} & \Xi_{12} \\
			\Xi_{12}^T & \Xi_{22}
		\end{bmatrix}^T -
		\begin{bmatrix}
			I & a \\
			0 & 1
		\end{bmatrix}
		\begin{bmatrix}
			\gamma_\infty^{-1}\mathcal{C}_q^T\mathcal{C}_q & \gamma_\infty^{-1}\mathcal{C}_q^T\mathcal{D}_{qp} \\
			\gamma_\infty^{-1}\mathcal{D}_{qp}^T\mathcal{C}_q & \gamma_\infty^{-1}\mathcal{D}_{qp}^T\mathcal{D}_{qp} - 1
		\end{bmatrix}
		\begin{bmatrix} 
			I & a \\
			0 & 1
		\end{bmatrix}^T
		\geq 0
	\end{equation*}

	which can be expanded into:
	
	\begin{multline} \label{eq:a-5}
		-\begin{bmatrix}
			\Xi_{11} & \Xi_{12} \\
			\Xi_{12}^T & \Xi_{22}
		\end{bmatrix} -
		\begin{bmatrix}
			I & a \\
			0 & 1
		\end{bmatrix}
		\begin{bmatrix}
			I & a\mathcal{D}_{qp}^T \\
			0 & \mathcal{D}_{qp}^T
		\end{bmatrix}
		\begin{bmatrix}
			\mathcal{C}_q^T \\
			1
		\end{bmatrix}
		\gamma_\infty^{-1}
		\begin{bmatrix}
			\mathcal{C}_q^T \\
			1
		\end{bmatrix}^T
		\begin{bmatrix}
			I & a\mathcal{D}_{qp}^T \\
			0 & \mathcal{D}_{qp}^T
		\end{bmatrix}^T
		\begin{bmatrix} 
			I & a \\
			0 & 1
		\end{bmatrix}^T + \\
		+\begin{bmatrix}
			I & a \\
			0 & 1
		\end{bmatrix}
		\begin{bmatrix}
			0 & 0 \\
			0 & 1
		\end{bmatrix}
		\begin{bmatrix} 
			I & a \\
			0 & 1
		\end{bmatrix}^T
		\geq 0.
	\end{multline}
	
	The 3 outer factors multiplied with $\gamma_\infty^{-1}$ in the middle term of \autoref{eq:a-5} are then combined together and the last summation term is also multiplied through, resulting in the following:
	
	\begin{equation} \label{eq:a-6}
		-\begin{bmatrix}
			\Xi_{11} & \Xi_{12} \\
			\Xi_{12}^T & \Xi_{22}
		\end{bmatrix}
		-\begin{bmatrix}
			\mathcal{C}_q^T + a\mathcal{D}_{qp}^T \\
			\mathcal{D}_{qp}^T
		\end{bmatrix}
		\gamma_\infty^{-1}
		\begin{bmatrix}
			\mathcal{C}_q^T + a\mathcal{D}_{qp}^T \\
			\mathcal{D}_{qp}^T
		\end{bmatrix}^T +
		\begin{bmatrix}
			aa^T & a \\
			a^T & 1
		\end{bmatrix}
		\geq 0.
	\end{equation}
	
	The last summation term of \autoref{eq:a-6} is then added with the linear part $\Xi$. Because $\gamma_\infty > 0 \leftrightarrow \gamma_\infty^{-1} > 0$, a Schur complement taken around $\gamma_\infty$ allows \autoref{eq:a-6} to be written as the single matrix inequality shown in \autoref{eq:lmiinf-equiv}.

\end{proof}

\section{Derivation of $\mathcal{H}_2$ and $\ell_1$ Inequalities}
\label{sec:apd-a-2}

\begin{thm}
	\autoref{eq:lmi2-1} from \autoref{sec:opt-h2} and \autoref{eq:lmi1-1} from \autoref{sec:opt-l1} are equivalent to the following:
	
	\begin{equation} \label{eq:lmi2-lmi1-equiv}
		\begin{bmatrix}
			-\Xi_{11} + aa^T & -\Xi_{12} + a \\
			-\Xi_{12}^T + a^T & -\Xi_{22} + 1
		\end{bmatrix} \geq 0,
	\end{equation}
	
	where \autoref{eq:lmi2-lmi1-equiv} contains just one nonlinear term in variable $a$, and:
	\vspace{-0.25cm} % To correct for there being one line on a new page at the end of A.2.

	\begin{equation*}
		\begin{bmatrix}
			\Xi_{11} & \Xi_{12} \\
			\Xi_{12}^T & \Xi_{22}
		\end{bmatrix} =
		\begin{bmatrix}
			I & a \\
			0 & 1
		\end{bmatrix}
		\begin{bmatrix}
			\mathcal{A} & \mathcal{B}_p \\
			I & 0
		\end{bmatrix}^T
		f\left(\Phi, P_\gamma, \alpha\right)
		\begin{bmatrix}
			\mathcal{A} & \mathcal{B}_p \\
			I & 0
		\end{bmatrix} 
		\begin{bmatrix}
			I & a \\
			0 & 1
		\end{bmatrix}^T
	\end{equation*}

	\begin{gather} \label{eq:pq}
		f\left(\Phi, P_\gamma, \alpha\right) = 
		\begin{cases}
			\Phi \oplus P_\gamma & \textrm{for the $\mathcal{H}_2$ case} \\
			\left(\Phi + \begin{bmatrix} 0 & 0 \\ 0 & \alpha \end{bmatrix}\right) \oplus P_\gamma & \textrm{for the $\ell_1$ case}
		\end{cases} \\
		P_\gamma = \gamma_\infty^{-1}P.
	\end{gather}
\end{thm}

\begin{proof}
	Starting from either \autoref{eq:lmi2-1} or \autoref{eq:lmi1-1}, the procedure mentioned in \autoref{sec:opt-h2l1} is followed to eliminate non-convex products in the first term of the \gls{LMI} independent of $f\left(\Phi, P_\gamma, \alpha\right)$. The second summation term is the same in both \gls{LMI}s and simplifies to \autoref{eq:cv-3}. Combining these, the matrix inequality from \autoref{eq:lmi2-lmi1-equiv} is produced.
\end{proof}