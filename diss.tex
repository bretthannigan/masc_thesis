%%%%%%%%%%%%%%%%%%%%%%%%%%%%%%%%%%%%%%%%%%%%%%%%%%%%%%%%%%%%%%%%%%%%%%
% Template for a UBC-compliant dissertation
% At the minimum, you will need to change the information found
% after the "Document meta-data"
%
%!TEX TS-program = pdflatex
%!TEX encoding = UTF-8 Unicode

%% The ubcdiss class provides several options:
%%   gpscopy (aka fogscopy)
%%       set parameters to exactly how GPS specifies
%%         * single-sided
%%         * page-numbering starts from title page
%%         * the lists of figures and tables have each entry prefixed
%%           with 'Figure' or 'Table'
%%       This can be tested by `\ifgpscopy ... \else ... \fi'
%%   10pt, 11pt, 12pt
%%       set default font size
%%   oneside, twoside
%%       whether to format for single-sided or double-sided printing
%%   balanced
%%       when double-sided, ensure page content is centred
%%       rather than slightly offset (the default)
%%   singlespacing, onehalfspacing, doublespacing
%%       set default inter-line text spacing; the ubcdiss class
%%       provides \textspacing to revert to this configured spacing
%%   draft
%%       disable more intensive processing, such as including
%%       graphics, etc.
%%

% For submission to GPS
\documentclass[gpscopy,onehalfspacing,11pt]{ubcdiss/ubcdiss}

% For your own copies (looks nicer)
% \documentclass[balanced,twoside,11pt]{ubcdiss}

%%%%%%%%%%%%%%%%%%%%%%%%%%%%%%%%%%%%%%%%%%%%%%%%%%%%%%%%%%%%%%%%%%%%%%
%%%%%%%%%%%%%%%%%%%%%%%%%%%%%%%%%%%%%%%%%%%%%%%%%%%%%%%%%%%%%%%%%%%%%%
%%
%% FONTS:
%% 
%% The defaults below configures Times Roman for the serif font,
%% Helvetica for the sans serif font, and Courier for the
%% typewriter-style font.  Configuring fonts can be time
%% consuming; we recommend skipping to END FONTS!
%% 
%% If you're feeling brave, have lots of time, and wish to use one
%% your platform's native fonts, see the commented out bits below for
%% XeTeX/XeLaTeX.  This is not for the faint at heart. 
%% (And shouldn't you be writing? :-)
%%

%% NFSS font specification (New Font Selection Scheme)
\usepackage{times,mathptmx,courier}
\usepackage[scaled=.92]{helvet}

%% Math or theory people may want to include the handy AMS macros
%\usepackage{amssymb}
%\usepackage{amsmath}
%\usepackage{amsfonts}

%% The pifont package provides access to the elements in the dingbat font.   
%% Use \ding{##} for a particular dingbat (see p7 of psnfss2e.pdf)
%%   Useful:
%%     51,52 different forms of a checkmark
%%     54,55,56 different forms of a cross (saltyre)
%%     172-181 are 1-10 in open circle (serif)
%%     182-191 are 1-10 black circle (serif)
%%     192-201 are 1-10 in open circle (sans serif)
%%     202-211 are 1-10 in black circle (sans serif)
%% \begin{dinglist}{##}\item... or dingautolist (which auto-increments)
%% to create a bullet list with the provided character.
\usepackage{pifont}

%%%%%%%%%%%%%%%%%%%%%%%%%%%%%%%%%%%%%%%%%%%%%%%%%%%%%%%%%%%%%%%%%%%%%%
%% Configure fonts for XeTeX / XeLaTeX using the fontspec package.
%% Be sure to check out the fontspec documentation.
%\usepackage{fontspec,xltxtra,xunicode}	% required
%\defaultfontfeatures{Mapping=tex-text}	% recommended
%% Minion Pro and Myriad Pro are shipped with some versions of
%% Adobe Reader.  Adobe representatives have commented that these
%% fonts can be used outside of Adobe Reader.
%\setromanfont[Numbers=OldStyle]{Minion Pro}
%\setsansfont[Numbers=OldStyle,Scale=MatchLowercase]{Myriad Pro}
%\setmonofont[Scale=MatchLowercase]{Andale Mono}

%% Other alternatives:
%\setromanfont[Mapping=tex-text]{Adobe Caslon}
%\setsansfont[Scale=MatchLowercase]{Gill Sans}
%\setsansfont[Scale=MatchLowercase,Mapping=tex-text]{Futura}
%\setmonofont[Scale=MatchLowercase]{Andale Mono}
%\newfontfamily{\SYM}[Scale=0.9]{Zapf Dingbats}
%% END FONTS
%%%%%%%%%%%%%%%%%%%%%%%%%%%%%%%%%%%%%%%%%%%%%%%%%%%%%%%%%%%%%%%%%%%%%%
%%%%%%%%%%%%%%%%%%%%%%%%%%%%%%%%%%%%%%%%%%%%%%%%%%%%%%%%%%%%%%%%%%%%%%



%%%%%%%%%%%%%%%%%%%%%%%%%%%%%%%%%%%%%%%%%%%%%%%%%%%%%%%%%%%%%%%%%%%%%%
%%%%%%%%%%%%%%%%%%%%%%%%%%%%%%%%%%%%%%%%%%%%%%%%%%%%%%%%%%%%%%%%%%%%%%
%%
%% Recommended packages
%%
\usepackage{checkend}	% better error messages on left-open environments
\usepackage{graphicx}	% for incorporating external images

%% booktabs: provides some special commands for typesetting tables as used
%% in excellent journals.  Ignore the examples in the Lamport book!
\usepackage{booktabs}

%% listings: useful support for including source code listings, with
%% optional special keyword formatting.  The \lstset{} causes
%% the text to be typeset in a smaller sans serif font, with
%% proportional spacing.
\usepackage{listings}
\lstset{basicstyle=\sffamily\scriptsize,showstringspaces=false,fontadjust}

%% The acronym package provides support for defining acronyms, providing
%% their expansion when first used, and building glossaries.  See the
%% example in glossary.tex and the example usage throughout the example
%% document.
%% NOTE: to use \MakeTextLowercase in the \acsfont command below,
%%   we *must* use the `nohyperlinks' option -- it causes errors with
%%   hyperref otherwise.  See Section 5.2 in the ``LaTeX 2e for Class
%%   and Package Writers Guide'' (clsguide.pdf) for details.
\usepackage[printonlyused,nohyperlinks]{acronym}
%% The ubcdiss.cls loads the `textcase' package which provides commands
%% for upper-casing and lower-casing text.  The following causes
%% the acronym package to typeset acronyms in small-caps
%% as recommended by Bringhurst.
\renewcommand{\acsfont}[1]{{\scshape \MakeTextLowercase{#1}}}

%% color: add support for expressing colour models.  Grey can be used
%% to great effect to emphasize other parts of a graphic or text.
%% For an excellent set of examples, see Tufte's "Visual Display of
%% Quantitative Information" or "Envisioning Information".
\usepackage{color}
\definecolor{greytext}{gray}{0.5}

%% comment: provides a new {comment} environment: all text inside the
%% environment is ignored.
%%   \begin{comment} ignored text ... \end{comment}
\usepackage{comment}

%% The natbib package provides more sophisticated citing commands
%% such as \citeauthor{} to provide the author names of a work,
%% \citet{} to produce an author-and-reference citation,
%% \citep{} to produce a parenthetical citation.
%% We use \citeeg{} to provide examples
\usepackage[numbers,sort&compress]{natbib}
\newcommand{\citeeg}[1]{\citep[e.g.,][]{#1}}

%% The titlesec package provides commands to vary how chapter and
%% section titles are typeset.  The following uses more compact
%% spacings above and below the title.  The titleformat that follow
%% ensure chapter/section titles are set in singlespace.
\usepackage[compact]{titlesec}
\titleformat*{\section}{\singlespacing\raggedright\bfseries\Large}
\titleformat*{\subsection}{\singlespacing\raggedright\bfseries\large}
\titleformat*{\subsubsection}{\singlespacing\raggedright\bfseries}
\titleformat*{\paragraph}{\singlespacing\raggedright\itshape}

%% The caption package provides support for varying how table and
%% figure captions are typeset.
\usepackage[format=hang,indention=-1cm,labelfont={bf},margin=1em]{caption}

%% url: for typesetting URLs and smart(er) hyphenation.
%% \url{http://...} 
\usepackage{url}
\urlstyle{sf}	% typeset urls in sans-serif


%%%%%%%%%%%%%%%%%%%%%%%%%%%%%%%%%%%%%%%%%%%%%%%%%%%%%%%%%%%%%%%%%%%%%%
%%%%%%%%%%%%%%%%%%%%%%%%%%%%%%%%%%%%%%%%%%%%%%%%%%%%%%%%%%%%%%%%%%%%%%
%%
%% Possibly useful packages: you may need to explicitly install
%% these from CTAN if they aren't part of your distribution;
%% teTeX seems to ship with a smaller base than MikTeX and MacTeX.
%%
%\usepackage{pdfpages}	% insert pages from other PDF files
%\usepackage{longtable}	% provide tables spanning multiple pages
%\usepackage{chngpage}	% support changing the page widths on demand
%\usepackage{tabularx}	% an enhanced tabular environment

%% enumitem: support pausing and resuming enumerate environments.
%\usepackage{enumitem}

%% rotating: provides two environments, sidewaystable and sidewaysfigure,
%% for typesetting tables and figures in landscape mode.  
%\usepackage{rotating}

%% subfig: provides for including subfigures within a figure,
%% and includes being able to separately reference the subfigures.
%\usepackage{subfig}

%% ragged2e: provides several new new commands \Centering, \RaggedLeft,
%% \RaggedRight and \justifying and new environments Center, FlushLeft,
%% FlushRight and justify, which set ragged text and are easily
%% configurable to allow hyphenation.
%\usepackage{ragged2e}

%% The ulem package provides a \sout{} for striking out text and
%% \xout for crossing out text.  The normalem and normalbf are
%% necessary as the package messes with the emphasis and bold fonts
%% otherwise.
%\usepackage[normalem,normalbf]{ulem}    % for \sout

%%%%%%%%%%%%%%%%%%%%%%%%%%%%%%%%%%%%%%%%%%%%%%%%%%%%%%%%%%%%%%%%%%%%%%
%% HYPERREF:
%% The hyperref package provides for embedding hyperlinks into your
%% document.  By default the table of contents, references, citations,
%% and footnotes are hyperlinked.
%%
%% Hyperref provides a very handy command for doing cross-references:
%% \autoref{}.  This is similar to \ref{} and \pageref{} except that
%% it automagically puts in the *type* of reference.  For example,
%% referencing a figure's label will put the text `Figure 3.4'.
%% And the text will be hyperlinked to the appropriate place in the
%% document.
%%
%% Generally hyperref should appear after most other packages

%% The following puts hyperlinks in very faint grey boxes.
%% The `pagebackref' causes the references in the bibliography to have
%% back-references to the citing page; `backref' puts the citing section
%% number.  See further below for other examples of using hyperref.
%% 2009/12/09: now use `linktocpage' (Jacek Kisynski): GPS now prefers
%%   that the ToC, LoF, LoT place the hyperlink on the page number,
%%   rather than the entry text.
\usepackage[bookmarks,bookmarksnumbered,%
    allbordercolors={0.8 0.8 0.8},%
    pagebackref,linktocpage%
    ]{hyperref}
%% The following change how the the back-references text is typeset in a
%% bibliography when `backref' or `pagebackref' are used
%%
%% Change \nocitations if you'd like some text shown where there
%% are no citations found (e.g., pulled in with \nocite{xxx})
\newcommand{\nocitations}{\relax}
%%\newcommand{\nocitations}{No citations}
%%
%\renewcommand*{\backref}[1]{}% necessary for backref < 1.33
\renewcommand*{\backrefsep}{,~}%
\renewcommand*{\backreftwosep}{,~}% ', and~'
\renewcommand*{\backreflastsep}{,~}% ' and~'
\renewcommand*{\backrefalt}[4]{%
\textcolor{greytext}{\ifcase #1%
\nocitations%
\or
\(\rightarrow\) page #2%
\else
\(\rightarrow\) pages #2%
\fi}}


%% The following uses most defaults, which causes hyperlinks to be
%% surrounded by colourful boxes; the colours are only visible in
%% PDFs and don't show up when printed:
%\usepackage[bookmarks,bookmarksnumbered]{hyperref}

%% The following disables the colourful boxes around hyperlinks.
%\usepackage[bookmarks,bookmarksnumbered,pdfborder={0 0 0}]{hyperref}

%% The following disables all hyperlinking, but still enabled use of
%% \autoref{}
%\usepackage[draft]{hyperref}

%% The following commands causes chapter and section references to
%% uppercase the part name.
\renewcommand{\chapterautorefname}{Chapter}
\renewcommand{\sectionautorefname}{Section}
\renewcommand{\subsectionautorefname}{Section}
\renewcommand{\subsubsectionautorefname}{Section}

%% If you have long page numbers (e.g., roman numbers in the 
%% preliminary pages for page 28 = xxviii), you might need to
%% uncomment the following and tweak the \@pnumwidth length
%% (default: 1.55em).  See the tocloft documentation at
%% http://www.ctan.org/tex-archive/macros/latex/contrib/tocloft/
% \makeatletter
% \renewcommand{\@pnumwidth}{3em}
% \makeatother

%%%%%%%%%%%%%%%%%%%%%%%%%%%%%%%%%%%%%%%%%%%%%%%%%%%%%%%%%%%%%%%%%%%%%%
%%%%%%%%%%%%%%%%%%%%%%%%%%%%%%%%%%%%%%%%%%%%%%%%%%%%%%%%%%%%%%%%%%%%%%
%%
%% Some special settings that controls how text is typeset
%%
% \raggedbottom		% pages don't have to line up nicely on the last line
% \sloppy		% be a bit more relaxed in inter-word spacing
% \clubpenalty=10000	% try harder to avoid orphans
% \widowpenalty=10000	% try harder to avoid widows
% \tolerance=1000

%% And include some of our own useful macros
\input{ubcdiss/macros}

%%%%%%%%%%%%%%%%%%%%%%%%%%%%%%%%%%%%%%%%%%%%%%%%%%%%%%%%%%%%%%%%%%%%%%
%%%%%%%%%%%%%%%%%%%%%%%%%%%%%%%%%%%%%%%%%%%%%%%%%%%%%%%%%%%%%%%%%%%%%%
%%
%% Document meta-data: be sure to also change the \hypersetup information
%%

\title{On the Use of the \texttt{ubcdiss} Template}
%\subtitle{If you want a subtitle}

\author{Johnny Canuck}
\previousdegree{B. Basket Weaving, University of Illustrious Arts, 1991}
\previousdegree{M. Silly Walks, Another University, 1994}

% What is this dissertation for?
\degreetitle{Doctor of Philosophy}

\institution{The University of British Columbia}
\campus{Vancouver}

\faculty{The Faculty of XXX}
\department{Basket Weaving}
\submissionmonth{April}
\submissionyear{2192}

% details of your examining committee
\examiningcommittee{John Smith, Materials Engineering}{Supervisor}
\examiningcommittee{Mary Maker, Materials Engineering}%
    {Supervisory Committee Member}
\examiningcommittee{Nebulous Name, Department}{Supervisory Committee Member}
\examiningcommittee{Magnus Monolith, Other Department}{Additional Examiner}

% details of your supervisory committee
\supervisorycommittee{Ira Crater, Materials Engineering}%
    {Supervisory Committee Member}
\supervisorycommittee{Adeline Long, \textsc{CEO} of Aerial Machine
    Transportation, Inc.}{Supervisory Committee Member}

%% hyperref package provides support for embedding meta-data in .PDF
%% files
\hypersetup{
  pdftitle={Change this title!  (DRAFT: \today)},
  pdfauthor={Johnny Canuck},
  pdfkeywords={Your keywords here}
}

%%%%%%%%%%%%%%%%%%%%%%%%%%%%%%%%%%%%%%%%%%%%%%%%%%%%%%%%%%%%%%%%%%%%%%
%%%%%%%%%%%%%%%%%%%%%%%%%%%%%%%%%%%%%%%%%%%%%%%%%%%%%%%%%%%%%%%%%%%%%%
%% 
%% The document content
%%

%% LaTeX's \includeonly commands causes any uses of \include{} to only
%% include files that are in the list.  This is helpful to produce
%% subsets of your thesis (e.g., for committee members who want to see
%% the dissertation chapter by chapter).  It also saves time by 
%% avoiding reprocessing the entire file.
%\includeonly{intro,conclusions}
%\includeonly{discussion}

\begin{document}

%%%%%%%%%%%%%%%%%%%%%%%%%%%%%%%%%%%%%%%%%%%%%%%%%%
%% From Thesis Components: Tradtional Thesis
%% <http://www.grad.ubc.ca/current-students/dissertation-thesis-preparation/order-components>

% Preliminary Pages (numbered in lower case Roman numerals)
%    1. Title page (mandatory)
\maketitle

%    2. Committee page (mandatory): lists supervisory committee and,
%    if applicable, the examining committee
\makecommitteepage

%    3. Abstract (mandatory - maximum 350 words)
%% The following is a directive for TeXShop to indicate the main file
%%!TEX root = diss.tex

\chapter{Abstract}

The sigma delta architecture of \gls{A/D} converters is especially applicable to digitizing most bio-signals. High order single-bit sigma delta modulators provide high resolution and linearity with low circuit complexity but require careful design to avoid unstable states. Many existing methods of designing these systems have few degrees of freedom, rely extensively on simulations, and do not provide guarantees about stability. The problem of designing sigma delta modulators with high performance and a clear indicator of the performance versus stability is addressed in this dissertation. This is done by developing a model of the sigma delta modulator that more accurately represents the system including robustness against the nonlinearities due to the quantizer element. After introducing many stability criteria from literature, those most suited to design are identified and the model adjusted to allow these criteria to be applied. High performance is maintained by using the \gls{GKYP} lemma to maximize noise rejection in the signal band using a \gls{SDP} framework that also permits the use of \gls{Hinf}, \gls{H2}, and \gls{l1} norm-based stability constraints on the system. Several designs using this framework are presented and their relative merits discussed. Examples include an aggressive noise shaping design to compete with existing methods on the basis of performance and designs with guaranteed stability for a range of input signals. The performance-stability trade-off for the different stability constraints using this work is examined and motivated by simulation results.

% Consider placing version information if you circulate multiple drafts
\vfill
\begin{center}
\begin{sf}
\fbox{Revision: \today}
\end{sf}
\end{center}

\cleardoublepage

%    4. Lay Summary (Effective May 2017, mandatory - maximum 150 words)
%% The following is a directive for TeXShop to indicate the main file
%%!TEX root = diss.tex

%% https://www.grad.ubc.ca/current-students/dissertation-thesis-preparation/preliminary-pages
%% 
%% LAY SUMMARY Effective May 2017, all theses and dissertations must
%% include a lay summary.  The lay or public summary explains the key
%% goals and contributions of the research/scholarly work in terms that
%% can be understood by the general public. It must not exceed 150
%% words in length.

\chapter{Lay Summary}

The lay or public summary explains the key goals and contributions of
the research\slash{}scholarly work in terms that can be understood by the
general public. It must not exceed 150 words in length.

\cleardoublepage

%    5. Preface
%% The following is a directive for TeXShop to indicate the main file
%%!TEX root = diss.tex

\chapter{Preface}

The work presented herein is an original independent production of the author. A manuscript has been submitted that contains a condensed version of much of the material in \autoref{ch:Modelling} and \autoref{ch:Optimization} as well as portions of \autoref{ch:Stability} and the \glsentrylong{CT} design from \autoref{sec:ex-ct}.

\cleardoublepage

%    6. Table of contents (mandatory - list all items in the preliminary pages
%    starting with the abstract, followed by chapter headings and
%    subheadings, bibliographies and appendices)
\tableofcontents
\cleardoublepage	% required by tocloft package

%    7. List of tables (mandatory if thesis has tables)
\listoftables
\cleardoublepage	% required by tocloft package

%    8. List of figures (mandatory if thesis has figures)
\listoffigures
\cleardoublepage	% required by tocloft package

%    9. List of illustrations (mandatory if thesis has illustrations)
%   10. Lists of symbols, abbreviations or other (optional)

%   11. Glossary (optional)
%% The following is a directive for TeXShop to indicate the main file
%%!TEX root = diss.tex

% Acronyms
\newacronym[type=hidden]{UBC}{UBC}{University of British Columbia}
\newacronym[type=hidden]{GPS}{GPS}{graduate and postdoctoral studies}

\newacronym{A/D}{A/D}{analog-to-digital}
\newacronym{D/A}{D/A}{digital-to-analog}
\newacronym{CT}{CT}{continuous-time}
\newacronym{DT}{DT}{discrete-time}
\newacronym{S/H}{S/H}{sample-and-hold}

\newacronym{EEG}{EEG}{electroencephalography}
\newacronym{ECG}{ECG}{electrocardiography}
\newacronym{PPG}{PPG}{photoplethysmography}

\newacronym{OSR}{OSR}{oversampling ratio}
\newacronym{SQNR}{SQNR}{signal-to-quantization-noise ratio}

\newacronym{AAF}{AAF}{antialiasing filter}
\newacronym{LF}{LF}{loop filter}
\newacronym{DRF}{DRF}{digital reconstruction filter}
\newacronym{LF}{LF}{loop filter}

\newacronym[description={noise transfer function, equivalent to the sensitivity function}]{NTF}{NTF}{noise transfer function}
\newacronym[description={signal transfer function, equivalent to the complementary sensitivity function}]{STF}{STF}{signal transfer function}

\newacronym{CLANS}{CLANS}{closed-loop analysis of noise shaper}
\newacronym{LMI}{LMI}{linear matrix inequality}
\newacronym{GKYP}{GKYP}{generalized Kalman-Yakubovi\v{c}-Popov}
\newacronym{FIR}{FIR}{finite impulse response}
\newacronym{IIR}{IIR}{infinite impulse response}
\newacronym{DSP}{DSP}{digital signal processing}


% Symbols
\newglossaryentry{pq}{
	name = $P_Q$ ,
	description = In-band quantization noise power
}
\newglossaryentry{delta}{
	name = $\Delta$ ,
	description = Quantization step size
}
\newglossaryentry{order}{
	name = $n$,
	description = Filter order
}
\newglossaryentry{r}{
	name = $r$,
	description = Analog reference input signal
}
\newglossaryentry{e}{
	name = $e$,
	description = Feedback error signal
}
\newglossaryentry{u}{
	name = $u$,
	description = Quantizer input signal
}
\newglossaryentry{y}{
	name = $y$,
	description = Digital bitstream output signal
}
\newglossaryentry{d}{
	name = $d$,
	description = Quantization noise source (linearized model)
}
\newglossaryentry{K}{
	name = $K$,
	description = Variable quantizer gain (linearized model)
}
\newglossaryentry{S}{
	name = $S(\lambda)$,
	description = Sensitivity function
}
\newglossaryentry{T}{
	name = $T(\lambda)$,
	description = Complementary sensitivity function
}
\newglossaryentry{sorz}{
	name = $\lambda$,
	description = Placeholder for the continuous-time Laplace variable $s$ or discrete-time $z$-transformation variable $z$
}

\renewcommand{\glsnamefont}[1]{{\scshape \MakeTextLowercase{#1}}}
\printglossary[title=List of Symbols]
\printglossary[type=\acronymtype,title=Glossary]

% You can also use \newacro{}{} to only define acronyms
% but without explictly creating a glossary
% 
% \newacro{ANOVA}[ANOVA]{Analysis of Variance\acroextra{, a set of
%   statistical techniques to identify sources of variability between groups.}}
% \newacro{API}[API]{application programming interface}
% \newacro{GOMS}[GOMS]{Goals, Operators, Methods, and Selection\acroextra{,
%   a framework for usability analysis.}}
% \newacro{TLX}[TLX]{Task Load Index\acroextra{, an instrument for gauging
%   the subjective mental workload experienced by a human in performing
%   a task.}}
% \newacro{UI}[UI]{user interface}
% \newacro{UML}[UML]{Unified Modelling Language}
% \newacro{W3C}[W3C]{World Wide Web Consortium}
% \newacro{XML}[XML]{Extensible Markup Language}
	% always input, since other macros may rely on it

\textspacing		% begin one-half or double spacing

%   12. Acknowledgements (optional)
%% The following is a directive for TeXShop to indicate the main file
%%!TEX root = diss.tex

\chapter{Acknowledgments}

I would like to thank my supervisor Prof. Guy Dumont for his support of my research as well as the other members of the BC Children's Hospital Research Institute's Digital Health Innovation Lab team.

I gratefully acknowledge the funding recieved from industry partner ESS Technologies whose participation in the Mitacs Accelerate program allowed me to write this thesis. The staff at ESS have been especially helpful, many thanks to Martin Mallinson and Chris Petersen in particular whose enthusiastic technical guidance and anecdoes about electronics, mathematics, and computing were a source of inspiration. Finally, thank you to the others who attended my progress meetings and provided valuable feedback.


%   13. Dedication (optional)

% Body of Thesis (not all sections may apply)
\mainmatter

\acresetall	% reset all acronyms used so far

%    1. Introduction
%% The following is a directive for TeXShop to indicate the main file
%%!TEX root = diss.tex

\chapter{Introduction}
\label{ch:Introduction}

The conversion of signals between analog and digital domanis is an often encountered problem in signal processing. For an analog signal to be represented digitally, it must undergo the processes of sampling and quantization. The former is the conversion from \gls{CT} to \gls{DT} and can be done without loss of information by the Nyquist-Shannon sampling theorem, given a sufficiently high sample rate. The latter is the mapping from an infinite set of possible values to a finite number of quantization levels. Unlike sampling, the process of quantization is non-injective and thus irreversible. The design of signal conversion circuits that minimize the error introduced by quantization is a major problem in mixed signal electronics.

\begin{figure}
	\centering
	\begin{tikzpicture}
	
		\node[coordinate] (w) at (0,0) {};
		\node[coordinate] (n) at (5,1.5) {};
		\node[coordinate] (s) at (5,-1.5) {};
		\node[coordinate] (e) at (8.5,0) {};
		\node[dspsquare] (sh2) at (4,0) {S/H};
		\node[dspsquare] (q2) at (6,0) {\RaisingEdge};
		\node[coordinate] (w-up) at ($(w) + (1,0.5)$) {};
		\node[coordinate] (w-right) at ($(w) + (1.25,0)$) {};
		\node[coordinate] (w-low) at ($(w) + (1,-0.5)$) {};
		\node[coordinate] (n-left) at ($(n) + (-1.5,0)$) {};
		\node[coordinate] (s-left) at  ($(s) + (-1.5,0)$) {};
		\node[coordinate] (n-right) at ($(n) + (1.25,0)$) {};
		\node[coordinate] (s-right) at ($(s) + (1.25,0)$) {};
		\node[coordinate] (mid) at ($(e) + (-1.5,0)$) {};
		\node[coordinate] (e-left) at ($(e) + (-1.5,0)$) {};
		
		\path (w-up) -- (n-left) node[dspsquare,midway,sloped] (sh1) {S/H};
		\path (w-up) -- (n-left) node[midway,sloped,yshift=20] {Sampling};
		\path (w-low) -- (s-left) node[dspsquare,midway,sloped] (q1) {\RaisingEdge};
		\path (w-low) -- (s-left) node[midway,sloped,yshift=-20] {Quantization};
		
		\draw[conndspconn] (w-up) -- (sh1);
		\draw[dspconn] (w-low) -- (q1);
		\draw[conndspconn] (sh1) -- (n-left);
		\draw[dspconn] (q1) -- (s-left);		
		\draw[conndspconn] (w-right) -- (sh2);
		\draw[conndspconn] (sh2) -- (q2);
		\draw[dspconn] (q2) -- (mid);
		
		% Analog signal.
		\begin{axis}[
			at={(w)},
			anchor=center,
			grid=major, 
			x=14mm, y=7mm, 
			axis x line=center, axis y line*=left, 
			ymin=-1, ymax=1, 
			yticklabels={}, xticklabels={},
			xlabel={$t$}, ylabel={$r(t)$},
			every axis x label/.style={
				at={(ticklabel* cs:1.05)},
				anchor=west
				},
			every axis y label/.style={
				at={(ticklabel* cs:0.5)},
				anchor=south,
				rotate=90
				},
			]
		\addplot[domain=-0:1,  smooth,  black]
		  plot {sin(deg(2*pi*1.33*x))};
		\end{axis}
		
		% Sampled plot.
		\begin{axis}[
			at={(n)},
			anchor=center,
			grid=major, 
			x=14mm, y=7mm, 
			axis x line=center, axis y line*=left, 
			ymin=-1, ymax=1, 
			yticklabels={}, xticklabels={}, 
			xlabel={$k$}, ylabel={$r[kT_s]$},
			every axis x label/.style={
				at={(ticklabel* cs:1.05)},
				anchor=west
				},
			every axis y label/.style={
				at={(ticklabel* cs:0.5)},
				anchor=south,
				rotate=90
				},
			]
		\addplot[ycomb, domain=-0:1,  samples=11, solid,  black, mark=*, mark size=1.5pt] 
		  plot {sin(deg(2*pi*1.33*x))};
		\end{axis}
		
		% Quantized plot.
		\begin{axis}[
			at={(s)},
			anchor=center,
			grid=major, 
			x=14mm, y=7mm, 
			axis x line=center, axis y line*=left, 
			ymin=-1, ymax=1, 
			yticklabels={}, xticklabels={}, 
			xlabel={$t$}, ylabel={$r_q(t)$},
			every axis x label/.style={
				at={(ticklabel* cs:1.05)},
				anchor=west
				},
			every axis y label/.style={
				at={(ticklabel* cs:0.5)},
				anchor=south,
				rotate=90
				},
			]
		\addplot[domain=-0:1, black]
		 coordinates {(0,0) (0.030,0) (0.030,0.5) (0.101,0.5) (0.101,1) (0.274,1) (0.274,0.5) (0.345,0.5) (0.345,0) (0.406,0) (0.406,-0.5) (0.477,-0.5) (0.477,-1) (0.650,-1) (0.650,-0.5) (0.721,-0.5) (0.721,0) (0.782,0) (0.782,0.5) (0.853,0.5) (0.853,1) (1,1)};
		 \end{axis}
		
		% Sampled and quantized plot.
		\begin{axis}[
			at={(e)},
			anchor=center,
			grid=major, 
			x=14mm, y=7mm, 
			axis x line=center, axis y line*=left, 
			ymin=-1, ymax=1, 
			yticklabels={}, xticklabels={}, 
			xlabel={$k$}, ylabel={$r_q[kT_s]$},
			every axis x label/.style={
				at={(ticklabel* cs:1.05)},
				anchor=west
				},
			every axis y label/.style={
				at={(ticklabel* cs:0.5)},
				anchor=south,
				rotate=90
				},
			]
		\addplot[ycomb, domain=-0:1, solid, black, mark=*, mark size=1.5pt]
		 coordinates {(0,0) (0.1,0.5) (0.2,1) (0.3,0.5) (0.4,0) (0.5,-1) (0.6,-1) (0.7,-0.5) (0.8,0.5) (0.9,1) (1,1)};
		\end{axis}
	\end{tikzpicture}
	\caption{A continuous-time, continuous-value signal $r(t)$ is sampled to produce a discrete-time, continuous-value signal $r[kT_s]$. $r(t)$ independently undergoes quantization to yield a continuous-time, discrete-value signal $r_q(t)$. When both processes are applied in sequence, a discrete-time, discrete-value signal $r_q[kT_s]$ is the result.}
	\label{fig:samp-quant}
\end{figure}

Sigma delta modulation is a widely used technique for \gls{A/D} and \gls{D/A} conversion of signals that provides high resolution through the techniques of oversampling and noise shaping. Oversampling trades throughput for resolution, thus the sigma delta modulator generally lies between integrating converters, which are specialized for near-dc signals, and high-speed architectures, such as successive approximation and flash. The sigma delta quantization scheme is especially applicable to signals with low to moderate frequency content. Signals with these properties include most biosignals such as those recorded electrically (\gls{EEG}, \gls{ECG}) or through other means using transducers (\gls{PPG}), as well as audio signals.

\section{Oversampling and Noise Shaping}
\label{sec:os-ns}

Oversampling is simply the process where the analog signal is sampled at a rate higher than what the sampling theorem would dictate for perfect reconstruction, expressed as the \gls{OSR} relative to the Nyquist frequency. It may seem that this does not have a direct benefit \emph{per se}, but it allows a less demanding analog \gls{AAF} to be used, saving circuit area. It also permits the quantization error to be spread across a larger bandwidth, reducing the in-band quantization noise power by a factor directly proportional to \gls{OSR}. These two advantages --- reducing analog circuit complexity and increasing resolution --- are common goals in sigma delta modulator design.

It may appear that oversampling alone quickly becomes impractical because one must approach very high sampling frequencies to increase the \gls{SQNR} substantially. However, this assumes that the quantization noise is evenly distributed across the spectrum. Noise shaping is the use of a filter operating on the oversampled signal to push quantization noise out of the signal band implemented by wrapping the quantizer in a feedback loop. 

\begin{figure}
\noindent\makebox[\textwidth]{
	\centering
	\begin{tikzpicture}[ampersand replacement=\&,scale=0.75, every node/.style={scale=0.75}]
		\node[coordinate] (g1) at (-2,2.5) {};
		% Place nodes using a matrix
		\matrix (m0) at (0,5) [row sep=0mm, column sep=5mm, matrix anchor=north west]
		{
			%--------------------------------------------------------------------------------------------------------------------------------
			\node[coordinate]						(m0-00) {};							\&
			\node[coordinate]						(m0-01) {};							\&
			\node[coordinate,label={above:$f_s$}]		(m0-02) {};							\& \\
			%--------------------------------------------------------------------------------------------------------------------------------
			\node[coordinate]						(m0-10) {};							\&
			\node[dspsquare,label={above:AAF}]			(m0-11) {};							\&
			\node[dspsquare]						(m0-12) {S/H};						\&
			\node[dspsquare]						(m0-13) {\RaisingEdge};					\& \\
			%--------------------------------------------------------------------------------------------------------------------------------
		};
		
		\draw[->] (m0-02) -- (m0-12);
		\draw[dspconn] (m0-10) -- (m0-11);
		\draw[dspconn] (m0-11) -- (m0-12);
		\draw[dspconn] (m0-12) -- (m0-13);
		\pgfplotsset{width=2.25cm,height=2.25cm,xmin=0.1,xmax=1000000,ymin=0.0001,ymax=10}
		\begin{loglogaxis}[
				at={($(m0-11) + (-9,-9)$)},
				ticks=none,
				axis x line*=bottom,
				axis y line*=left
			 ]
			\addplot[domain=1:100000]  {(60*x+10000)/(x*x + 60*x+10000)};
		\end{loglogaxis}
		
		\matrix (m1) at (0,2.5) [row sep=0mm, column sep=5mm, matrix anchor=north west]
		{
			%--------------------------------------------------------------------------------------------------------------------------------
			\node[coordinate]						(m1-00) {};							\&
			\node[coordinate]						(m1-01) {};							\&
			\node[coordinate,label={above:$OSR\cdot f_s$}]	(m1-02) {};							\& \\
			%--------------------------------------------------------------------------------------------------------------------------------
			\node[dspnodefull]						(m1-10) {};							\&
			\node[dspsquare,label={above:AAF}]			(m1-11) {};							\&
			\node[dspsquare]						(m1-12) {S/H};						\&
			\node[dspsquare]						(m1-13) {\RaisingEdge};					\&
			\node[dspsquare,label={above:DRF}]			(m1-14) {};							\&
			\node[dspfilter]						(m1-15) {$\downarrow OSR$};				\& \\
			%--------------------------------------------------------------------------------------------------------------------------------
		};
		
		\draw[->] (m1-02) -- (m1-12);
		\draw[dspconn] (m1-10) -- (m1-11);
		\draw[dspconn] (m1-11) -- (m1-12);
		\draw[dspconn] (m1-12) -- (m1-13);
		\draw[dspconn] (m1-13) -- (m1-14);
		\draw[dspconn] (m1-14) -- (m1-15);
		\pgfplotsset{width=2.25cm,height=2.25cm,xmin=0.1,xmax=1000000,ymin=0.0001,ymax=10}
		\begin{loglogaxis}[
				at={($(m1-11) + (-9,-9)$)},
				ticks=none,
				axis x line*=bottom,
				axis y line*=left
			 ]
			\addplot[domain=1:100000]  {(60*x+10000)/(x*x + 60*x+10000)};
		\end{loglogaxis}
		\pgfplotsset{width=2.25cm,height=2.25cm,xmin=0.1,xmax=1000000,ymin=0.0001,ymax=10}
		\begin{loglogaxis}[
				at={($(m1-14) + (-9,-9)$)},
				ticks=none,
				axis x line*=bottom,
				axis y line*=left
			 ]
			\addplot[domain=1:100000]  {(60*x+10000)/(x*x + 60*x+10000)};
		\end{loglogaxis}
		
		\matrix (m2) at (0,0) [row sep=0mm, column sep=5mm, matrix anchor=north west]
		{
			%--------------------------------------------------------------------------------------------------------------------------------
			\node[coordinate]						(m2-00) {};							\&
			\node[coordinate]						(m2-01) {};							\&
			\node[coordinate,label={above:$OSR\cdot f_s$}]	(m2-02) {};							\& \\
			%--------------------------------------------------------------------------------------------------------------------------------
			\node[coordinate]						(m2-10) {};							\&
			\node[dspsquare,label={above:AAF}]			(m2-11) {};							\&
			\node[dspsquare]						(m2-12) {S/H};						\&
			\node[dspadder,label={below left:$-$}]		(m2-13) {};							\&
			\node[dspsquare,label={above:LF}]			(m2-14) {$\int$};						\&
			\node[dspsquare]						(m2-15) {\RaisingEdge};					\&
			\node[dspnodefull]						(m2-16) {};							\&
			\node[dspsquare,label={above:DRF}]			(m2-17) {};							\&
			\node[dspfilter]						(m2-18) {$\downarrow OSR$};				\&
			\node[dspnodeopen,dsp/label=right]			(m2-19) {$y_3[k]$};						\& \\
			%--------------------------------------------------------------------------------------------------------------------------------
			\node[coordinate]						(m2-20) {};							\&
			\node[coordinate]						(m2-21) {};							\&
			\node[coordinate]						(m2-22) {};							\&
			\node[coordinate]						(m2-23) {};							\&
			\node[coordinate]						(m2-24) {};							\&
			\node[coordinate]						(m2-25) {};							\&
			\node[coordinate]						(m2-26) {};							\& \\
		};
		
		\draw[->] (m2-02) -- (m2-12);
		\draw[dspconn] (m2-10) -- (m2-11);
		\draw[dspconn] (m2-11) -- (m2-12);
		\draw[dspconn] (m2-12) -- (m2-13);
		\draw[dspconn] (m2-13) -- (m2-14);
		\draw[dspconn] (m2-14) -- (m2-15);
		\draw[dspline] (m2-15) -- (m2-16);
		\draw[dspconn] (m2-16) -- (m2-17); 
		\draw[dspconn] (m2-17) -- (m2-18);
		\draw[dspconn] (m2-18) -- (m2-19);
		\draw[dspline] (m2-16) -- (m2-26);
		\draw[dspline] (m2-26) -- (m2-23);
		\draw[dspconn] (m2-23) -- (m2-13);
		\pgfplotsset{width=2.25cm,height=2.25cm,xmin=0.1,xmax=1000000,ymin=0.0001,ymax=10}
		\begin{loglogaxis}[
				at={($(m2-11) + (-9,-9)$)},
				ticks=none,
				axis x line*=bottom,
				axis y line*=left
			 ]
			\addplot[domain=1:100000]  {(60*x+10000)/(x*x + 60*x+10000)};
		\end{loglogaxis}
		\pgfplotsset{width=2.25cm,height=2.25cm,xmin=0.1,xmax=1000000,ymin=0.0001,ymax=10}
		\begin{loglogaxis}[
				at={($(m2-17) + (-9,-9)$)},
				ticks=none,
				axis x line*=bottom,
				axis y line*=left
			 ]
			\addplot[domain=1:100000]  {(60*x+10000)/(x*x + 60*x+10000)};
		\end{loglogaxis}
		
		\draw[dspline] (m0-10) -- (m1-10);
		\draw[dspline] (m1-10) -- (m2-10);
		\node[dspnodeopen,dsp/label=left] (r) at ($(m1-10) + (-0.5cm, 0)$) {$r(t)$};
		\draw[dspline] (r) -- (m1-10);
		
		\node[coordinate] (g-y2) at ($(m2-19) + (4cm,0)$) {};
		\coordinate (g-y1) at (g-y2 |- m1-15);
		\coordinate (g-y0) at (g-y2 |- m0-13);
		\node[dspnodeopen,dsp/label=right] (y1) at (m2-19 |- m1-15) {$y_2[k]$};
		\node[dspnodeopen,dsp/label=right] (y0) at (m2-19 |- m0-13) {$y_1[k]$};
		
		\draw[dspconn] (m0-13) -- (y0);
		\draw[dspconn] (m1-15) -- (y1);
		
		\begin{axis}[
			at={($(r) + (-3cm,0)$)},
			anchor=center,
			width=6cm, height=3.75cm,
			anchor=center, 
			xmin=0, xmax=4,
			ymin=-100, ymax=300,
			axis x line=bottom, axis y line=left, axis line style={-},
			xticklabels={0,1,2,3,\SI{4}{\second}}, xtick={0,1,2,3,4},
			yticklabels={, 0, \SI{300}{\micro\volt}}, ytick={-100, 0, 300}
			]
			\addplot[solid,black] table [x=t, y=r, col sep=comma] {data/comparison-r.csv};
		\end{axis}
		
		\begin{axis}[
			at={(g-y0)},
			anchor=center,
			width=6cm, height=3.75cm,
			anchor=center, 
			xmin=0, xmax=4,
			ymin=-100, ymax=300,
			axis x line=none, axis y line=left, axis line style={-},
			xticklabels={0,1,2,3,\SI{4}{\second}}, xtick={0,1,2,3,4},
			yticklabels={, 0, \SI{300}{\micro\volt}}, ytick={-100, 0, 300}
			]
			\addplot[solid,black] table [x=t, y=y, col sep=comma] {data/comparison-y0.csv};
		\end{axis}
		
		\begin{axis}[
			at={(g-y1)},
			anchor=center,
			width=6cm, height=3.75cm,
			anchor=center, 
			xmin=0, xmax=4,
			ymin=-100, ymax=300,
			axis x line=none, axis y line=left, axis line style={-},
			xticklabels={0,1,2,3,\SI{4}{\second}}, xtick={0,1,2,3,4},
			yticklabels={, 0, \SI{300}{\micro\volt}}, ytick={-100, 0, 300}
			]
			\addplot[solid,black] table [x=t, y=y, col sep=comma] {data/comparison-y1.csv};
		\end{axis}
		
		\begin{axis}[
			at={(g-y2)},
			anchor=center,
			width=6cm, height=3.75cm,
			anchor=center, 
			xmin=0, xmax=4,
			ymin=-100, ymax=300,
			axis x line=bottom, axis y line=left, axis line style={-},
			xticklabels={0,1,2,3,\SI{4}{\second}}, xtick={0,1,2,3,4},
			yticklabels={, 0, \SI{300}{\micro\volt}}, ytick={-100, 0, 300}
			]
			\addplot[solid,black] table [x=t, y=y, col sep=comma] {data/comparison-y2.csv};
		\end{axis}
		
	\end{tikzpicture}}
	\caption{A comparison between na\"{i}ve quantization (top), 10 times oversampled quantization (middle), and first order sigma delta modulation (bottom). The graphs on the right show the increasing quality of an \gls{EEG} signal sampled to a final rate of \SI{100}{\hertz} and quantized to 5 bits by each scheme.}
	\label{fig:samp-quant}
\end{figure}

\section{Related Works}

\section{Organization of this Thesis}
\label{sec:org}

\gls{angelsperarea}

This document provides a quick set of instructions for using the
\class{ubcdiss} class to write a dissertation in \LaTeX. 
Unfortunately this document cannot provide an introduction to using
\LaTeX.  The classic reference for learning \LaTeX\ is
\citeauthor{lamport-1994-ladps}'s
book~\cite{lamport-1994-ladps}.  There are also many freely-available
tutorials online;
\webref{http://www.andy-roberts.net/misc/latex/}{Andy Roberts' online
    \LaTeX\ tutorials}
seems to be excellent.
The source code for this docment, however, is intended to serve as
an example for creating a \LaTeX\ version of your dissertation.

We start by discussing organizational issues, such as splitting
your dissertation into multiple files, in
\autoref{sec:SuggestedThesisOrganization}.
We then cover the ease of managing cross-references in \LaTeX\ in
\autoref{sec:CrossReferences}.
We cover managing and using bibliographies with \BibTeX\ in
\autoref{sec:BibTeX}. 
We briefly describe typesetting attractive tables in
\autoref{sec:TypesettingTables}.
We briefly describe including external figures in
\autoref{sec:Graphics}, and using special characters and symbols
in \autoref{sec:SpecialSymbols}.
As it is often useful to track different versions of your dissertation,
we discuss revision control further in
\autoref{sec:DissertationRevisionControl}. 
We conclude with pointers to additional sources of information in
\autoref{sec:Conclusions}.

%%%%%%%%%%%%%%%%%%%%%%%%%%%%%%%%%%%%%%%%%%%%%%%%%%%%%%%%%%%%%%%%%%%%%%
\section{Suggested Thesis Organization}
\label{sec:SuggestedThesisOrganization}

The specifies a particular arrangement of the
components forming a thesis.\footnote{See
    \url{http://www.grad.ubc.ca/current-students/dissertation-thesis-preparation/order-components}}
This template reflects that arrangement.

In terms of writing your thesis, the recommended best practice for
organizing large documents in \LaTeX\ is to place each chapter in
a separate file.  These chapters are then included from the main
file through the use of \verb+\include{file}+.  A thesis might
be described as six files such as \file{intro.tex},
\file{relwork.tex}, \file{model.tex}, \file{eval.tex},
\file{discuss.tex}, and \file{concl.tex}.

We also encourage you to use macros for separating how something
will be typeset (\eg bold, or italics) from the meaning of that
something. 
For example, if you look at \file{intro.tex}, you will see repeated
uses of a macro \verb+\file{}+ to indicate file names.
The \verb+\file{}+ macro is defined in the file \file{macros.tex}.
The consistent use of \verb+\file{}+ throughout the text not only
indicates that the argument to the macro represents a file (providing
meaning or semantics), but also allows easily changing how
file names are typeset simply by changing the definition of the
\verb+\file{}+ macro.
\file{macros.tex} contains other useful macros for properly typesetting
things like the proper uses of the latinate \emph{exempli grati\={a}}
and \emph{id est} (\ie \verb+\eg+ and \verb+\ie+), 
web references with a footnoted \acs{URL} (\verb+\webref{url}{text}+),
as well as definitions specific to this documentation
(\verb+\latexpackage{}+).

%%%%%%%%%%%%%%%%%%%%%%%%%%%%%%%%%%%%%%%%%%%%%%%%%%%%%%%%%%%%%%%%%%%%%%
\section{Making Cross-References}
\label{sec:CrossReferences}

\LaTeX\ make managing cross-references easy, and the \latexpackage{hyperref}
package's\ \verb+\autoref{}+ command\footnote{%
    The \latexpackage{hyperref} package is included by default in this
    template.}
makes it easier still. 

A thing to be cross-referenced, such as a section, figure, or equation,
is \emph{labelled} using a unique, user-provided identifier, defined
using the \verb+\label{}+ command.  
The thing is referenced elsewhere using the \verb+\autoref{}+ command.
For example, this section was defined using:
\begin{lstlisting}
    \section{Making Cross-References}
    \label{sec:CrossReferences}
\end{lstlisting}
References to this section are made as follows:
\begin{lstlisting}
    We then cover the ease of managing cross-references in \LaTeX\
    in \autoref{sec:CrossReferences}.
\end{lstlisting}
\verb+\autoref{}+ takes care of determining the \emph{type} of the 
thing being referenced, so the example above is rendered as
\begin{quote}
    We then cover the ease of managing cross-references in \LaTeX\
    in \autoref{sec:CrossReferences}.
\end{quote}

The label is any simple sequence of characters, numbers, digits,
and some punctuation marks such as ``:'' and ``--''; there should
be no spaces.  Try to use a consistent key format: this simplifies
remembering how to make references.  This document uses a prefix
to indicate the type of the thing being referenced, such as \texttt{sec}
for sections, \texttt{fig} for figures, \texttt{tbl} for tables,
and \texttt{eqn} for equations.

For details on defining the text used to describe the type
of \emph{thing}, search \file{diss.tex} and the \latexpackage{hyperref}
documentation for \texttt{autorefname}.


%%%%%%%%%%%%%%%%%%%%%%%%%%%%%%%%%%%%%%%%%%%%%%%%%%%%%%%%%%%%%%%%%%%%%%
\section{Managing Bibliographies with \BibTeX}
\label{sec:BibTeX}

One of the primary benefits of using \LaTeX\ is its companion program,
\BibTeX, for managing bibliographies and citations.  Managing
bibliographies has three parts: (i) describing references,
(ii)~citing references, and (iii)~formatting cited references.

\subsection{Describing References}

\BibTeX\ defines a standard format for recording details about a
reference.  These references are recorded in a file with a
\file{.bib} extension.  \BibTeX\ supports a broad range of
references, such as books, articles, items in a conference proceedings,
chapters, technical reports, manuals, dissertations, and unpublished
manuscripts. 
A reference may include attributes such as the authors,
the title, the page numbers, t.  A reference
can also be augmented with personal attributes, such as a rating,
notes, or keywords.

Each reference must be described by a unique \emph{key}.\footnote{%
    Note that the citation keys are different from the reference
    identifiers as described in \autoref{sec:CrossReferences}.}
A key is a simple sequence of characters, numbers, digits, and some
punctuation marks such as ``:'' and ``--''; there should be no spaces. 
A consistent key format simiplifies remembering how to make references. 
For example:
\begin{quote}
   \fbox{\emph{last-name}}\texttt{-}\fbox{\emph{year}}\texttt{-}\fbox{\emph{contracted-title}}
\end{quote}
where \emph{last-name} represents the last name for the first author,
and \emph{contracted-title} is some meaningful contraction of the
title.  Then \citeauthor{kiczales-1997-aop}'s seminal article on
aspect-oriented programming~\cite{kiczales-1997-aop} (published in
\citeyear{kiczales-1997-aop}) might be given the key
\texttt{kiczales-1997-aop}.

An example of a \BibTeX\ \file{.bib} file is included as
\file{biblio.bib}.  A description of the format a \file{.bib}
file is beyond the scope of this document.  We instead encourage
you to use one of the several reference managers that support the
\BibTeX\ format such as
\webref{http://jabref.sourceforge.net}{JabRef} (multiple platforms) or
\webref{http://bibdesk.sourceforge.net}{BibDesk} (MacOS\,X only). 
These front ends are similar to reference manages such as
EndNote or RefWorks.


\subsection{Citing References}

Having described some references, we then need to cite them.  We
do this using a form of the \verb+\cite+ command.  For example:
\begin{lstlisting}
    \citet{kiczales-1997-aop} present examples of crosscutting 
    from programs written in several languages.
\end{lstlisting}
When processed, the \verb+\citet+ will cause the paper's authors
and a standardized reference to the paper to be inserted in the
document, and will also include a formatted citation for the paper
in the bibliography.  For example:
\begin{quote}
    \citet{kiczales-1997-aop} present examples of crosscutting 
    from programs written in several languages.
\end{quote}
There are several forms of the \verb+\cite+ command (provided
by the \latexpackage{natbib} package), as demonstrated in
\autoref{tbl:natbib:cite}.
Note that the form of the citation (numeric or author-year) depends
on the bibliography style (described in the next section).
The \verb+\citet+ variant is used when the author names form
an object in the sentence, whereas the \verb+\citep+ variant
is used for parenthetic references, more like an end-note.
Use \verb+\nocite+ to include a citation in the bibliography
but without an actual reference.
\nocite{rowling-1997-hpps}
\begin{table}
\caption{Available \texttt{cite} variants; the exact citation style
    depends on whether the bibliography style is numeric or author-year.}
\label{tbl:natbib:cite}
\centering
\begin{tabular}{lp{3.25in}}\toprule
Variant & Result \\
\midrule
% We cheat here to simulate the cite/citep/citet for APA-like styles
\verb+\cite+ & Parenthetical citation (\eg ``\cite{kiczales-1997-aop}''
    or ``(\citeauthor{kiczales-1997-aop} \citeyear{kiczales-1997-aop})'') \\
\verb+\citet+ & Textual citation: includes author (\eg
    ``\citet{kiczales-1997-aop}'' or
    or ``\citeauthor{kiczales-1997-aop} (\citeyear{kiczales-1997-aop})'') \\
\verb+\citet*+ & Textual citation with unabbreviated author list \\
\verb+\citealt+ & Like \verb+\citet+ but without parentheses \\
\verb+\citep+ & Parenthetical citation (\eg ``\cite{kiczales-1997-aop}''
    or ``(\citeauthor{kiczales-1997-aop} \citeyear{kiczales-1997-aop})'') \\
\verb+\citep*+ & Parenthetical citation with unabbreviated author list \\
\verb+\citealp+ & Like \verb+\citep+ but without parentheses \\
\verb+\citeauthor+ & Author only (\eg ``\citeauthor{kiczales-1997-aop}'') \\
\verb+\citeauthor*+ & Unabbreviated authors list 
    (\eg ``\citeauthor*{kiczales-1997-aop}'') \\
\verb+\citeyear+ & Year of citation (\eg ``\citeyear{kiczales-1997-aop}'') \\
\bottomrule
\end{tabular}
\end{table}

\subsection{Formatting Cited References}

\BibTeX\ separates the citing of a reference from how the cited
reference is formatted for a bibliography, specified with the
\verb+\bibliographystyle+ command. 
There are many varieties, such as \texttt{plainnat}, \texttt{abbrvnat},
\texttt{unsrtnat}, and \texttt{vancouver}.
This document was formatted with \texttt{abbrvnat}.
Look through your \TeX\ distribution for \file{.bst} files. 
Note that use of some \file{.bst} files do not emit all the information
necessary to properly use \verb+\citet{}+, \verb+\citep{}+,
\verb+\citeyear{}+, and \verb+\citeauthor{}+.

There are also packages available to place citations on a per-chapter
basis (\latexpackage{bibunits}), as footnotes (\latexpackage{footbib}),
and inline (\latexpackage{bibentry}).
Those who wish to exert maximum control over their bibliography
style should see the amazing \latexpackage{custom-bib} package.

%%%%%%%%%%%%%%%%%%%%%%%%%%%%%%%%%%%%%%%%%%%%%%%%%%%%%%%%%%%%%%%%%%%%%%
\section{Typesetting Tables}
\label{sec:TypesettingTables}

\citet{lamport-1994-ladps} made one grievous mistake
in \LaTeX: his suggested manner for typesetting tables produces
typographic abominations.  These suggestions have unfortunately
been replicated in most \LaTeX\ tutorials.  These
abominations are easily avoided simply by ignoring his examples
illustrating the use of horizontal and vertical rules (specifically
the use of \verb+\hline+ and \verb+|+) and using the
\latexpackage{booktabs} package instead.

The \latexpackage{booktabs} package helps produce tables in the form
used by most professionally-edited journals through the use of
three new types of dividing lines, or \emph{rules}.
% There are times that you don't want to use \autoref{}
Tables~\ref{tbl:natbib:cite} and~\ref{tbl:LaTeX:Symbols} are two
examples of tables typeset with the \latexpackage{booktabs} package.
The \latexpackage{booktabs} package provides three new commands
for producing rules:
\verb+\toprule+ for the rule to appear at the top of the table,
\verb+\midrule+ for the middle rule following the table header,
and \verb+\bottomrule+ for the bottom-most at the end of the table.
These rules differ by their weight (thickness) and the spacing before
and after.
A table is typeset in the following manner:
\begin{lstlisting}
    \begin{table}
    \caption{The caption for the table}
    \label{tbl:label}
    \centering
    \begin{tabular}{cc}
    \toprule
    Header & Elements \\
    \midrule
    Row 1 & Row 1 \\
    Row 2 & Row 2 \\
    % ... and on and on ...
    Row N & Row N \\
    \bottomrule
    \end{tabular}
    \end{table}
\end{lstlisting}
See the \latexpackage{booktabs} documentation for advice in dealing with
special cases, such as subheading rules, introducing extra space
for divisions, and interior rules.

%%%%%%%%%%%%%%%%%%%%%%%%%%%%%%%%%%%%%%%%%%%%%%%%%%%%%%%%%%%%%%%%%%%%%%
\section{Figures, Graphics, and Special Characters}
\label{sec:Graphics}

Most \LaTeX\ beginners find figures to be one of the more challenging
topics.  In \LaTeX, a figure is a \emph{floating element}, to be
placed where it best fits.
The user is not expected to concern him/herself with the placement
of the figure.  The figure should instead be labelled, and where
the figure is used, the text should use \verb+\autoref+ to reference
the figure's label.
\autoref{fig:latex-affirmation} is an example of a figure.
\begin{figure}
    \centering
    % For the sake of this example, we'll just use text
    %\includegraphics[width=3in]{file}
    \Huge{\textsf{\LaTeX\ Rocks!}}
    \caption{Proof of \LaTeX's amazing abilities}
    \label{fig:latex-affirmation}   % label should change
\end{figure}
A figure is generally included as follows:
\begin{lstlisting}
    \begin{figure}
    \centering
    \includegraphics[width=3in]{file}
    \caption{A useful caption}
    \label{fig:fig-label}   % label should change
    \end{figure}
\end{lstlisting}
There are three items of note:
\begin{enumerate}
\item External files are included using the \verb+\includegraphics+
    command.  This command is defined by the \latexpackage{graphicx} package
    and can often natively import graphics from a variety of formats.
    The set of formats supported depends on your \TeX\ command processor.
    Both \texttt{pdflatex} and \texttt{xelatex}, for example, can
    import \textsc{gif}, \textsc{jpg}, and \textsc{pdf}.  The plain
    version of \texttt{latex} only supports \textsc{eps} files.

\item The \verb+\caption+ provides a caption to the figure. 
    This caption is normally listed in the List of Figures; you
    can provide an alternative caption for the LoF by providing
    an optional argument to the \verb+\caption+ like so:
    \begin{lstlisting}
    \caption[nice shortened caption for LoF]{%
	longer detailed caption used for the figure}
    \end{lstlisting}
    \ac{GPS} generally prefers shortened single-line captions
    in the LoF: multiple-line captions are a bit unwieldy.

\item The \verb+\label+ command provides for associating a unique, user-defined,
    and descriptive identifier to the figure.  The figure can be
    can be referenced elsewhere in the text with this identifier
    as described in \autoref{sec:CrossReferences}.
\end{enumerate}
See Keith Reckdahl’s excellent guide for more details,
\webref{http://www.ctan.org/tex-archive/info/epslatex.pdf}{\emph{Using
imported graphics in LaTeX2e}}.

\section{Special Characters and Symbols}
\label{sec:SpecialSymbols}

\LaTeX\ appropriates many common symbols for its own purposes,
with some used for commands (\ie \verb+\{}&%+) and
mathematics (\ie \verb+$^_+), and others are automagically transformed
into typographically-preferred forms (\ie \verb+-`'+) or to
completely different forms (\ie \verb+<>+).
\autoref{tbl:LaTeX:Symbols} presents a list of common symbols and
their corresponding \LaTeX\ commands.  A much more comprehensive list 
of symbols and accented characters is available at:
\url{http://www.ctan.org/tex-archive/info/symbols/comprehensive/}
\begin{table}
\caption{Useful \LaTeX\ symbols}\label{tbl:LaTeX:Symbols}
\centering\begin{tabular}{ccp{0.5cm}cc}\toprule
\LaTeX & Result && \LaTeX & Result \\
\midrule
    \verb+\texttrademark+ & \texttrademark && \verb+\&+ & \& \\
    \verb+\textcopyright+ & \textcopyright && \verb+\{ \}+ & \{ \} \\
    \verb+\textregistered+ & \textregistered && \verb+\%+ & \% \\
    \verb+\textsection+ & \textsection && \verb+\verb!~!+ & \verb!~! \\
    \verb+\textdagger+ & \textdagger && \verb+\$+ & \$ \\
    \verb+\textdaggerdbl+ & \textdaggerdbl && \verb+\^{}+ & \^{} \\
    \verb+\textless+ & \textless && \verb+\_+ & \_ \\
    \verb+\textgreater+ & \textgreater && \\
\bottomrule
\end{tabular}
\end{table}

%%%%%%%%%%%%%%%%%%%%%%%%%%%%%%%%%%%%%%%%%%%%%%%%%%%%%%%%%%%%%%%%%%%%%%
\section{Changing Page Widths and Heights}

The \class{ubcdiss} class is based on the standard \LaTeX\ \class{book}
class~\cite{lamport-1994-ladps} that selects a line-width to carry
approximately 66~characters per line.  This character density is
claimed to have a pleasing appearance and also supports more rapid
reading~\cite{bringhurst-2002-teots}.  I would recommend that you
not change the line-widths!

\subsection{The \texttt{geometry} Package}

Some students are unfortunately saddled with misguided supervisors
or committee members whom believe that documents should have the
narrowest margins possible.  The \latexpackage{geometry} package is
helpful in such cases.  Using this package is as simple as:
\begin{lstlisting}
    \usepackage[margin=1.25in,top=1.25in,bottom=1.25in]{geometry}
\end{lstlisting}
You should check the package's documentation for more complex uses.

\subsection{Changing Page Layout Values By Hand}

There are some miserable students with requirements for page layouts
that vary throughout the document.  Unfortunately the
\latexpackage{geometry} can only be specified once, in the document's
preamble.  Such miserable students must set \LaTeX's layout parameters
by hand:
\begin{lstlisting}
    \setlength{\topmargin}{-.75in}
    \setlength{\headsep}{0.25in}
    \setlength{\headheight}{15pt}
    \setlength{\textheight}{9in}
    \setlength{\footskip}{0.25in}
    \setlength{\footheight}{15pt}

    % The *sidemargin values are relative to 1in; so the following
    % results in a 0.75 inch margin
    \setlength{\oddsidemargin}{-0.25in}
    \setlength{\evensidemargin}{-0.25in}
    \setlength{\textwidth}{7in}       % 1.1in margins (8.5-2*0.75)
\end{lstlisting}
These settings necessarily require assuming a particular page height
and width; in the above, the setting for \verb+\textwidth+ assumes
a \textsc{US} Letter with an 8.5'' width.
The \latexpackage{geometry} package simply uses the page height and
other specified values to derive the other layout values.
The
\href{http://tug.ctan.org/tex-archive/macros/latex/required/tools/layout.pdf}{\texttt{layout}}
package provides a
handy \verb+\layout+ command to show the current page layout
parameters. 


\subsection{Making Temporary Changes to Page Layout}

There are occasions where it becomes necessary to make temporary
changes to the page width, such as to accomodate a larger formula. 
The \latexmiscpackage{chngpage} package provides an \env{adjustwidth}
environment that does just this.  For example:
\begin{lstlisting}
    % Expand left and right margins by 0.75in
    \begin{adjustwidth}{-0.75in}{-0.75in}
    % Must adjust the perceived column width for LaTeX to get with it.
    \addtolength{\columnwidth}{1.5in}
    \[ an extra long math formula \]
    \end{adjustwidth}
\end{lstlisting}


%%%%%%%%%%%%%%%%%%%%%%%%%%%%%%%%%%%%%%%%%%%%%%%%%%%%%%%%%%%%%%%%%%%%%%
\section{Keeping Track of Versions with Revision Control}
\label{sec:DissertationRevisionControl}

Software engineers have used \acf{RCS} to track changes to their
software systems for decades.  These systems record the changes to
the source code along with context as to why the change was required.
These systems also support examining and reverting to particular
revisions from their system's past.

An \ac{RCS} can be used to keep track of changes to things other
than source code, such as your dissertation.  For example, it can
be useful to know exactly which revision of your dissertation was
sent to a particular committee member.  Or to recover an accidentally
deleted file, or a badly modified image.  With a revision control
system, you can tag or annotate the revision of your dissertation
that was sent to your committee, or when you incorporated changes
from your supervisor.

Unfortunately current revision control packages are not yet targetted
to non-developers.  But the Subversion project's
\webref{http://tortoisesvn.net/docs/release/TortoiseSVN_en/}{TortoiseSVN}
has greatly simplified using the Subversion revision control system
for Windows users.  You should consult your local geek.

A simpler alternative strategy is to create a GoogleMail account
and periodically mail yourself zipped copies of your dissertation.

%%%%%%%%%%%%%%%%%%%%%%%%%%%%%%%%%%%%%%%%%%%%%%%%%%%%%%%%%%%%%%%%%%%%%%
\section{Recommended Packages}

The real strength to \LaTeX\ is found in the myriad of free add-on
packages available for handling special formatting requirements.
In this section we list some helpful packages.

\subsection{Typesetting}

\begin{description}
\item[\latexpackage{enumitem}:]
    Supports pausing and resuming enumerate environments.

\item[\latexpackage{ulem}:]
    Provides two new commands for striking out and crossing out text
    (\verb+\sout{text}+ and \verb+\xout{text}+ respectively)
    The package should likely
    be used as follows:
    \begin{verbatim}
    \usepackage[normalem,normalbf]{ulem}
    \end{verbatim}
    to prevent the package from redefining the emphasis and bold fonts.

\item[\latexpackage{chngpage}:]
    Support changing the page widths on demand.

\item[\latexpackage{mhchem}:] 
    Support for typesetting chemical formulae and reaction equations.

\end{description}

Although not a package, the
\webref{http://www.ctan.org/tex-archive/support/latexdiff/}{\texttt{latexdiff}}
command is very useful for creating changebar'd versions of your
dissertation.


\subsection{Figures, Tables, and Document Extracts}

\begin{description}
\item[\latexpackage{pdfpages}:]
    Insert pages from other PDF files.  Allows referencing the extracted
    pages in the list of figures, adding labels to reference the page
    from elsewhere, and add borders to the pages.

\item[\latexpackage{subfig}:]
    Provides for including subfigures within a figure, and includes
    being able to separately reference the subfigures.  This is a
    replacement for the older \texttt{subfigure} environment.

\item[\latexpackage{rotating}:]
    Provides two environments, sidewaystable and sidewaysfigure,
    for typesetting tables and figures in landscape mode.  

\item[\latexpackage{longtable}:]
    Support for long tables that span multiple pages.

\item[\latexpackage{tabularx}:]
    Provides an enhanced tabular environment with auto-sizing columns.

\item[\latexpackage{ragged2e}:]
    Provides several new commands for setting ragged text (\eg forms
    of centered or flushed text) that can be used in tabular
    environments and that support hyphenation.

\end{description}


\subsection{Bibliography Related Packages}

\begin{description}
\item[\latexpackage{bibunits}:]
    Support having per-chapter bibliographies.

\item[\latexpackage{footbib}:]
    Cause cited works to be rendered using footnotes.

\item[\latexpackage{bibentry}:] 
    Support placing the details of a cited work in-line.

\item[\latexpackage{custom-bib}:]
    Generate a custom style for your bibliography.

\end{description}


%%%%%%%%%%%%%%%%%%%%%%%%%%%%%%%%%%%%%%%%%%%%%%%%%%%%%%%%%%%%%%%%%%%%%%
\section{Moving On}
\label{sec:Conclusions}

At this point, you should be ready to go.  Other handy web resources:
\begin{itemize}
\item \webref{http://www.ctan.org}{\ac{CTAN}} is \emph{the} comprehensive
    archive site for all things related to \TeX\ and \LaTeX. 
    Should you have some particular requirement, somebody else is
    almost certainly to have had the same requirement before you,
    and the solution will be found on \ac{CTAN}.  The links to
    various packages in this document are all to \ac{CTAN}.

\item An online
    \webref{http://www.ctan.org/get/info/latex2e-help-texinfo/latex2e.html}{%
	reference to \LaTeX\ commands} provides a handy quick-reference
    to the standard \LaTeX\ commands.

\item The list of 
    \webref{http://www.tex.ac.uk/cgi-bin/texfaq2html?label=interruptlist}{%
	Frequently Asked Questions about \TeX\ and \LaTeX}
    can save you a huge amount of time in finding solutions to
    common problems.

\item The \webref{http://www.tug.org/tetex/tetex-texmfdist/doc/}{te\TeX\
    documentation guide} features a very handy list of the most useful
    packages for \LaTeX\ as found in \ac{CTAN}.

\item The
\webref{http://www.ctan.org/tex-archive/macros/latex/required/graphics/grfguide.pdf}{\texttt{color}}
    package, part of the graphics bundle, provides handy commands
    for changing text and background colours.  Simply changing
    text to various levels of grey can have a very 
    \textcolor{greytext}{dramatic effect}.


\item If you're really keen, you might want to join the
    \webref{http://www.tug.org}{\TeX\ Users Group}.

\end{itemize}

\endinput

Any text after an \endinput is ignored.
You could put scraps here or things in progress.


%    2. Main body
% Generally recommended to put each chapter into a separate file
%\include{relatedwork}
%%% The following is a directive for TeXShop to indicate the main file
%%!TEX root = diss.tex

\chapter{Modelling the Sigma Delta Modulator}
\label{ch:Modelling}

In order to apply an optimization framework to the design of the \gls{LF}, the system from Figures \ref{fig:basic-struct-dt} and \ref{fig:basic-struct-ct} must be placed in a form that allows tractable application of the desired performance and stability targets. This includes omission of blocks that have minimal or no effect on the loop as well as linearization of the quantizer. The \gls{AAF} (when present) can be considered as a pre-filter operating on the input signal. The filter $H_0(\lambda)$ serves as an additional degree of freedom for the \gls{STF} can be set to unity for the purposes of the model. These two filters are not required in stabilitiy analysis, because the \gls{NTF} depends only on $H_1(\lambda)$ as seen in \autoref{eq:t}. After noise rejection performance has been optimized, $H_0(\lambda)$ can be tuned as necessary to ensure that the combined gain of the \gls{AAF} and \gls{LF} is close to unity in the signal band. In a similar way, the \gls{DSP} in the output path serves only to filter out the signal and decimate to the original sampling frequency which may be dealt with separately without impacting loop stability.

\section{Linearization of the Quantizer Element}
\label{sec:model-linq}

Next, the nonlinear nature of the quantizer is dealt with. As mentioned before, a common linearization approach is to replace the quantizer with an additive noise source \gls{d}. Furthermore, the linear model can incorporate a variable gain \gls{K}. The inclusion of \gls{K} has uses in linearization, stability, and performance that will be expanded upon in \autoref{ch:Stability}. After these simplifications, the block diagram in \autoref{fig:sdm-model} is obtained, which is applicable to \gls{DT} or \gls{CT} designs. In the \gls{DT} case, the loop is operating entirely in the oversampled domain and the \gls{S/H} block is not shown. In the \gls{CT} case, the \gls{S/H} block in the loop is neglected so that \gls{S} and \gls{T} are \gls{CT} transfer functions\footnote{Note that regarding \autoref{fig:sdm-model} and Figures~\ref{fig:basic-struct-dt}/\ref{fig:basic-struct-ct}, the \gls{NTF} $S(\lambda)$ is the same transfer function whether interpreted from $d \rightarrow y$ or $r \rightarrow e$.}.

\begin{figure}
	\centering
	% General Sigma Delta Modulator
	\begin{tikzpicture}[ampersand replacement=\&,scale=0.75, every node/.style={scale=0.75}]
	
		% Place nodes using a matrix
		\matrix (m1) [row sep=2.5mm, column sep=5mm]
		{
			%-----------------------------------------------------------------------------------------------------------------------------------------------------------------------
			\node[coordinate]											(m01) {};							\&
			\node[coordinate]											(m02) {};							\&
			\node[coordinate]											(m03) {};							\&
			\node[coordinate]											(m04) {};							\&
			\node[dspnodeopen,dsp/label=above]								(m05) {$d$};							\& \\
			%-----------------------------------------------------------------------------------------------------------------------------------------------------------------------
			\node[dspnodeopen,dsp/label=left]								(m11) {$r$};							\&
			\node[dspadder,label={below left:$-$}]							(m12) {};							\&
			\node[dspfilter]											(m13) {$H_1(\lambda)$};					\&
			\node[dspgain,fill=white,label={[align=center,yshift=-5]below:Linearized\\Gain}]	(m14) {$K$};							\&
			\node[dspsquare,label={below:Quantizer}]							(m15) {\RaisingEdge};					\&
			\node[dspnodefull]											(m16) {};							\&
			\node[dspnodeopen,dsp/label=right]								(m17) {$y$};							\& \\
			%-----------------------------------------------------------------------------------------------------------------------------------------------------------------------
			\node[coordinate]											(m21) {};							\&
			\node[coordinate]											(m22) {};							\& 
			\node[coordinate]											(m23) {};							\& 
			\node[coordinate]											(m24) {};							\& 
			\node[coordinate]											(m25) {};							\& 
			\node[coordinate]											(m26) {};							\& 
			\node[coordinate]											(m27) {};							\& \\
			%-----------------------------------------------------------------------------------------------------------------------------------------------------------------------
		};
	
		\node[draw,inner xsep=15pt,inner ysep=10pt,dashed,fit={($(m05.north)+(-0.5, 0.7)$) ($(m15.south)+(0.5, -0.6)$)},label={[align=center]above:Linear Model}] {};
		%\node[draw,inner xsep=15pt,inner ysep=10pt,dashed,fit={($(m02.north west)+(-0.25, 0.25)$) ($(m13.south east)+(0.4, -0.5)$)},label=below:{Loop Filter $H(\lambda)$}] {};
	
		\begin{pgfonlayer}{bg}
			\draw[->]		($(m14) + (-0.5,-0.5)$) -- ($(m14) + (0.5,0.5)$);
		\end{pgfonlayer}
		\draw[dspconn] 	(m11) -- (m12);
		\draw[dspconn] 	(m12) -- (m13);
		\draw[dspconn] 	(m13) -- node[midway,above] {$u$} (m14);
		\draw[dspconn]	(m14) -- (m15);
		\draw[dspconn] 	($(m14.east)+(6pt, 0)$) -- (m15);
		\draw[dspline]	(m15) -- (m16);
		\draw[dspconn] 	(m16) -- (m17);
		\draw[dspline] 	(m16) -- (m26);
		\draw[dspline] 	(m26) -- (m22);
		\draw[dspconn]	(m22) -- (m12);
		\draw[dspconn] 	(m12) -- node[midway,above] {$e$} (m13);
		\draw[dspconn] 	(m05) -- (m15);
		\draw[Gray, ->, out=40, in=90, looseness=0.85] ($(m11)+(0, 0.4)$) to node[below, xshift=-22pt, yshift=-5pt] {$T(\lambda)$} ($(m17)+(0.35, 0.4)$);
		\draw[Gray, ->, out=-45, in=135, looseness=1] ($(m05)+(0.35, 0.25)$) to node[above, xshift=5pt] {$S(\lambda)$} ($(m17)+(0.25, 0.35)$);
	
	\end{tikzpicture}
	\caption{The linearized sigma delta loop block diagram with omission of extraneous filters and the quantizer replaced by a variable gain and additive quantization noise signal.}  \label{fig:sdm-model}
\end{figure}

\section{Well-Posedness and Internal Stability}
\label{sec:model-wp-is}

The meaningful application of feedback to reduce an uncertainty (in this case, error introduced by the nonlinear quantizer) requires that the system be well-posed in order for a solution to exist. \autoref{fig:sdm-model} can undergo block diagram mainpulation bringing it into the standard feedback form shown in \autoref{fig:sdm-stdf} with signals \gls{r}, \gls{e}, \gls{d}, and \gls{y}.

\begin{figure}[h]
	\centering
	% General Sigma Delta Modulator
	\begin{tikzpicture}[ampersand replacement=\&,scale=0.75, every node/.style={scale=0.75}]
	
		% Place nodes using a matrix
		\matrix (m1) [row sep=2.5mm, column sep=5mm]
		{
			%--------------------------------------------------------------------------------
			\node[coordinate]					(m00) {};		\&
			\node[coordinate]					(m01) {};		\&
			\node[dspfilter]					(m02) {$-1$};	\&
			\node[dspadder]					(m03) {};		\&
			\node[dspnodeopen,dsp/label=right]		(m04) {$r$};		\& \\
			%--------------------------------------------------------------------------------
			\node[dspnodeopen,dsp/label=left]		(m10) {$d$};		\&
			\node[dspadder]					(m11) {};		\&
			\node[dspfilter]					(m12) {$H_1K$};	\&
			\node[coordinate]					(m13) {};		\& \\
			%--------------------------------------------------------------------------------
		};
	
		\draw[dspconn]	(m01) -- (m02);
		\draw[dspconn] 	(m02) -- (m03);
		\draw[dspconn] 	(m04) -- (m03);
		\draw[dspline] 	(m01) to node[midway,left] {$y$} (m11);
		\draw[dspline]	(m03) to node[midway,right] {$e$} (m13);
		\draw[dspconn] 	(m10) -- (m11);
		\draw[dspconn]	(m12) -- (m11);
		\draw[dspconn]	(m13) -- (m12);
	
	\end{tikzpicture}
	\caption{The linearized model converted into standard feedback form.}  \label{fig:sdm-stdf}
\end{figure}

With some abuse of notation, the equations describing this loop are:

\begin{equation}
	\begin{bmatrix}
		r \\
		d
	\end{bmatrix} =
	\begin{bmatrix}
		1 & 1 \\
		-H_1K & 1
	\end{bmatrix}
	\begin{bmatrix}
		e \\
		y
	\end{bmatrix}. \label{eq:stdf}
\end{equation}

A feedback system is considered well-posed if the inverse of the transfer matrix in \autoref{eq:stdf} exists and each of its elements are proper. \autoref{eq:stdf-inv} shows that this is the case if both $S(\lambda)$ and $T(\lambda)$ are proper transfer functions.

\begin{equation}
	\begin{bmatrix}
		e \\
		y
	\end{bmatrix} =
	\begin{bmatrix}
		1 & 1 \\
		-H_1K & 1
	\end{bmatrix}^{-1} =
	\begin{bmatrix}
		\frac{1}{1 + H_1K} & \frac{-1}{1 + H_1K} \\
		\frac{H_1K}{1 + H_1K} & \frac{1}{1 + H_1K}
	\end{bmatrix}
	\begin{bmatrix}
		r \\
		d
	\end{bmatrix} = 
	\begin{bmatrix}
		S & -S \\
		T & S
	\end{bmatrix}
	\begin{bmatrix}
		r \\
		d
	\end{bmatrix}. \label{eq:stdf-inv}
\end{equation}

The principle of internal stability is stricter than \gls{BIBO} stability because it guarantees that the internal states of the system remain bounded. The system in \autoref{eq:stdf-inv} is internally stable if each element of the transfer matrix belongs to the set $\mathcal{R}\mathcal{H}_\infty$, i.e., the set of stable real rational proper transfer functions.

\subsection{Constraints on the \titlecap{\glsentrylong{NTF}}}
\label{sec:model-ntf-constraints}

A sufficient condition for $S(\lambda)$ and $T(\lambda)$ to be proper is that transfer function $H_1(\lambda)$ is a strictly proper real rational transfer function. Internal stability of the system follows if $S(\lambda)$ and $T(\lambda)$ are stable. This leads to the following constraints on the \gls{NTF}:

\begin{enumerate}
	\item $S(\lambda)$ is stable, and \label{it:con-1},
	\item The following equivalent conditions hold: \label{it:con-2}
		\begin{enumerate}
			\item $S(\infty) = 1$,
			\item If $S(\lambda)$ is placed in state-space form, the feedthrough matrix $D=1$, and,
			\item The first element of the impulse response of $S(\lambda)$ is one.
		\end{enumerate}
\end{enumerate}

Most prior work in the area performs optimization directly on the \gls{NTF} of the system. This is effective because it is a relatively accurate model of the noise shaping performance. In addition, constraint \ref{it:con-2} enforces causality on the feedback loop ensuring the system is physically realizable.

\section{Modelling Uncertain Quantizer Gain}
\label{sec:model-lft}

Having established conditions to ensure the closed-loop system is realizable and internally stable, there remains a nonlinear gain block $K$. $K$ can be understood as a time-varying gain dependent on the quantizer input. For example, a 1-bit quantizer (\gls{delta}$ = 2$) with output $\{-1, 1\}$ would have instantaneous gain $K(t) = \frac{1}{u(t)}$. As the value of $u$ at each sample time is not known in advance, $K$ may be modelled as a multiplicative uncertainty. The upper \gls{LFT} allows $K$ to be separated into a constant gain matrix $M_{2 \times 2}$ and a normalized, \gls{Hinf} norm-bounded uncertain block $\Delta$ by Expression~\ref{eq:lft}.

\begin{equation}
	K \leftrightarrow \mathcal{F}_U\{M, \Delta\} \quad ||\Delta||_\infty \leq 1 \label{eq:lft}
\end{equation}

The model from \autoref{fig:sdm-stdf} is shown in \autoref{fig:sdm-stdf-lft} with the quantizer and variable gain replaced by this \gls{LFT} interconnection. In \autoref{ch:Optimization}, it is of interest to ensure the robustness of the system to $\Delta$, which may be achieved using this model.

\begin{figure}[h]
	\centering
	% Augmented Plant
	\begin{tikzpicture}[ampersand replacement=\&,scale=0.75, every node/.style={scale=0.75}]
	
		% Place nodes using a matrix
		\matrix (m1) [row sep=2.5mm, column sep=5mm]
		{
			%-----------------------------------------------------------------------------------------------
			\node[coordinate]					(m00) {};				\&
			\node[coordinate]					(m01) {};				\&
			\node[coordinate]					(m03) {};				\&
			\node[dspsquare]					(m05) {$\Delta$};			\&
			\node[coordinate]					(m06) {};				\& \\
			%-----------------------------------------------------------------------------------------------
			\node[dspnodeopen,dsp/label=left]		(m10) {$r$};				\&
			\node[dspadder,label={below left:$-$}]	(m11) {};				\&
			\node[dspfilter]					(m13) {$H_1(\lambda)$};		\&
			\node[dspfilter,yshift=5]				(m15) {$M$};			\&
			\node[dspnodefull]					(m16) {};				\&
			\node[dspnodeopen,dsp/label=right]		(m17) {$y$};				\& \\
			%-----------------------------------------------------------------------------------------------
			\node[coordinate]					(m20) {};				\&
			\node[coordinate]					(m21) {};				\&
			\node[coordinate]					(m23) {};				\& 
			\node[coordinate]					(m25) {};				\& 
			\node[coordinate]					(m26) {};				\& \\
			%-----------------------------------------------------------------------------------------------
		};
		
		%\node[draw,inner xsep=15pt,inner ysep=10pt,dashed,fit={($(m10) + (0,0.6)$) ($(m16)+(0.85, -1)$)},label={[align=center]below:Augmented Plant $G(\lambda)$}] {};
	
		\draw[dspconn] (m10) to node[midway,above] {$e$} (m11);
		\draw[dspconn] (m11) to node[midway,above] {$u$} (m13);
		\node[coordinate] at ($(m15) + (-20pt,-5pt)$) (m15-sw) {};
		\node[coordinate] at ($(m15) + (20pt,-5pt)$) (m15-se) {};
		\node[coordinate] at ($(m15) + (-20pt,5pt)$) (m15-nw) {};
		\node[coordinate] at ($(m15) + (20pt,5pt)$) (m15-ne) {};
		\node[coordinate] at ($(m15-nw) + (-10pt,0)$) (m15-nw-corner) {};
		\node[coordinate] at ($(m15-ne) + (10pt,0)$) (m15-ne-corner) {};
		\node[coordinate] at (m15-nw-corner |- m05.west) (m05-w-corner) {};
		\node[coordinate] at (m15-ne-corner |- m05.east) (m05-e-corner) {};
		
		\draw[dspconn] (m13) -- (m15-sw);
		\draw[dspconn] (m15-se) -- (m17);
		\draw[dspconn] (m15-nw-corner) -- (m15-nw);
		\draw[dspline] (m15-nw-corner) -- (m05-w-corner);
		\draw[dspline] (m05) -- (m05-w-corner);
		\draw[dspline] (m15-ne-corner) -- (m15-ne);
		\draw[dspline] (m15-ne-corner) -- (m05-e-corner);
		\draw[dspconn] (m05-e-corner) -- (m05);
		\draw[dspline] (m16) -- (m26);
		\draw[dspline] (m26) -- (m21);
		\draw[dspconn] (m21) -- (m11);
		%\draw[black!20, ->, out=-90, in=-90, looseness=0.6] ($(m10)+(-0.35, -0.25)$) to node[below, xshift=-22pt] {$STF$} ($(m17)+(0.35, -0.25)$);
		%\draw[RedOrange, ->, out=90, in=180, looseness=1] ($(m10)+(-0.35, 0.25)$) to node[left, xshift=-2pt,yshift=2pt] {$NTF$} ($(m02)+(-0.35, 0.35)$);
		
	\end{tikzpicture}
	\caption{The linearized block diagram with the quantizer replaced by a multiplicative uncertainty extracted via \glsentryshort{LFT}.} \label{fig:sdm-stdf-lft}
\end{figure}

\section{Derivation of Augmented System}

\subsection{Extraction of Performance and Stability Channels}

Finally, the model is abstracted into an augmented form where all desired input and output channels are present and all unnecessary ones hidden. Let the \gls{LFT} $\Delta \rightarrow M$ input and $M \rightarrow \Delta$ output be \gls{w} and \gls{z}, respectively. These channels are required to be accessed in addition to \gls{r}, \gls{e}, \gls{u}, and \gls{y} for the purposes listed in \autoref{tab:aug-ch}. The augmented system \gls{G} is encapsulated as the dashed block in \autoref{fig:sdm-aug}.

\begin{table}[t]
	\centering
	\caption{Input and output channels of interest for the augmented system.} \label{tab:aug-ch}
	%\renewcommand{\arraystretch}{2}
	\begin{tabular}{>{\centering\arraybackslash}m{1.6cm} | >{\RaggedRight}m{2cm} >{\RaggedRight}m{2.25cm} >{\RaggedRight}m{2.25cm} >{\RaggedRight}m{2cm}}
		\toprule
		\diagbox[width=2cm, height=1cm]{\textbf{Input}}{\textbf{Output}} & \multicolumn{1}{c}{$z$} & \multicolumn{1}{c}{$e$} & \multicolumn{1}{c}{$u$} & \multicolumn{1}{c}{$y$} \\
		\midrule
		$r$ & Not used & \gls{NTF} performance channel & Constraint on quantizer input signal & \gls{STF} constraints for \gls{CT} design \\
		$w$ & Quantizer gain robustness channel & Not used & Not used & Not used \\
		\bottomrule
	\end{tabular}
\end{table}

\begin{figure}
	\centering
	% Augmented Plant
	\begin{tikzpicture}[ampersand replacement=\&,scale=0.75, every node/.style={scale=0.75}]
	
		% Place nodes using a matrix
		\matrix (m1) [row sep=2.5mm, column sep=5mm]
		{
			%-----------------------------------------------------------------------------------------------
			\node[coordinate]					(m00) {};				\&
			\node[coordinate]					(m01) {};				\&
			\node[dspnodeopen,dsp/label=above]		(m02) {$e$};				\&
			\node[coordinate]					(m03) {};				\&
			\node[dspnodeopen,dsp/label=above]		(m04) {$u$};				\&
			\node[dspsquare]					(m05) {$\Delta$};			\&
			\node[coordinate]					(m06) {};				\& \\
			%-----------------------------------------------------------------------------------------------
			\node[dspnodeopen,dsp/label=left]		(m10) {$r$};				\&
			\node[dspadder,label={below left:$-$}]	(m11) {};				\&
			\node[dspnodefull]					(m12) {};				\&
			\node[dspfilter]					(m13) {$H_1(\lambda)$};		\&
			\node[dspnodefull]					(m14) {};				\&
			\node[dspfilter,yshift=5]				(m15) {$M$};			\&
			\node[dspnodefull]					(m16) {};				\&
			\node[dspnodeopen,dsp/label=right]		(m17) {$y$};				\& \\
			%-----------------------------------------------------------------------------------------------
			\node[coordinate]					(m20) {};				\&
			\node[coordinate]					(m21) {};				\&
			\node[coordinate]					(m22) {};				\& 
			\node[coordinate]					(m23) {};				\& 
			\node[coordinate]					(m24) {};				\&
			\node[coordinate]					(m25) {};				\& 
			\node[coordinate]					(m26) {};				\& \\
			%-----------------------------------------------------------------------------------------------
		};
		
		\node[draw,inner xsep=15pt,inner ysep=10pt,dashed,fit={($(m10) + (0,0.6)$) ($(m16)+(0.85, -1)$)},label={[align=center]below:Augmented System $G(\lambda)$}] {};
	
		\draw[dspconn] (m10) -- (m11);
		\draw[dspconn] (m11) -- (m13);
		\draw[dspconn] (m12) -- (m02);
		\node[coordinate] at ($(m15) + (-20pt,-5pt)$) (m15-sw) {};
		\node[coordinate] at ($(m15) + (20pt,-5pt)$) (m15-se) {};
		\node[coordinate] at ($(m15) + (-20pt,5pt)$) (m15-nw) {};
		\node[coordinate] at ($(m15) + (20pt,5pt)$) (m15-ne) {};
		\node[coordinate] at ($(m15-nw) + (-10pt,0)$) (m15-nw-corner) {};
		\node[coordinate] at ($(m15-ne) + (10pt,0)$) (m15-ne-corner) {};
		\node[coordinate] at (m15-nw-corner |- m05.west) (m05-w-corner) {};
		\node[coordinate] at (m15-ne-corner |- m05.east) (m05-e-corner) {};
		
		\draw[dspconn] (m13) -- (m15-sw);
		\draw[dspconn] (m15-se) -- (m17);
		\draw[dspconn] (m15-nw-corner) -- (m15-nw);
		\draw[dspline] (m15-nw-corner) -- (m05-w-corner);
		\draw[dspline] (m05) -- node[midway,above] {$w$} (m05-w-corner);
		\draw[dspline] (m15-ne-corner) -- (m15-ne);
		\draw[dspline] (m15-ne-corner) -- (m05-e-corner);
		\draw[dspconn] (m05-e-corner) -- node[midway,above] {$z$} (m05);
		\draw[dspline] (m16) -- (m26);
		\draw[dspline] (m26) -- (m21);
		\draw[dspconn] (m21) -- (m11);
		\draw[dspconn] (m14) -- (m04);
		%\draw[black!20, ->, out=-90, in=-90, looseness=0.6] ($(m10)+(-0.35, -0.25)$) to node[below, xshift=-22pt] {$STF$} ($(m17)+(0.35, -0.25)$);
		%\draw[RedOrange, ->, out=90, in=180, looseness=1] ($(m10)+(-0.35, 0.25)$) to node[left, xshift=-2pt,yshift=2pt] {$NTF$} ($(m02)+(-0.35, 0.35)$);
		
	\end{tikzpicture}
	\caption{The augmented plant is derived by setting $H_0(\lambda) = 1$, taking the LFT of the uncertain gain, extracting the signals of interest, and writing the closed-loop equations.} \label{fig:sdm-aug}
\end{figure}

\subsection{Derivation of State-Space Model}

Now that the desired input and output signals are captured by the model, it is a simple exercise to write the system in state-space form. To begin, let filter $H_1(\lambda)$ be the transfer function of order \gls{order} in variable \gls{sorz}$ = z$ in the \gls{DT} case (or \gls{sorz}$ = s$ in the \gls{CT} case). The numerator and denominator coefficients are shown in \autoref{eq:h1} which has the equivalent state-space representation of \autoref{eq:h1-ss}.

\begin{align}
	H_1(\lambda) = \frac{U(\lambda)}{E(\lambda)} &= \frac{b_{n-1}\lambda^{n-1} + b_{n-2}\lambda^{n-2} + \ldots + b_1\lambda + b_0}{\lambda^n + a_{n-1}\lambda^{n-1} + a_{n-2}\lambda^{n-2} + \ldots + a_1z\lambda + a_0} \label{eq:h1} \\
	&= C_H(\lambda I - A_H)^{-1}B_H \label{eq:h1-ss}
\end{align}

Naturally, $H_1(\lambda)$ is a strictly proper transfer function and state-space feedthrough matrix $D_H = 0$ due to the constraints proposed in \autoref{sec:model-wp-is}. The constant gain matrix \gls{M} may be split into its constituent parts:

\begin{equation}
	M = 
	\begin{bmatrix}
		m_{11} & m_{12} \\
		m_{21} & m_{22}
	\end{bmatrix}. \label{eq:m-exp}
\end{equation}

With some algebra, the augmented system \gls{G} from \autoref{fig:sdm-aug} may be written in state-space form with notation from Equations~\ref{eq:h1-ss} and \ref{eq:m-exp} by introducing state vector $x$. The notation $G_{qp}(\lambda)$ is used to indicate the transfer function of \gls{G} from some input channel $p$ to some output channel $q$ and the closed-loop state-space matrix blocks are denoted with cursive symbols as shown in \autoref{eq:aug-ss}.

\begin{align}
	G:
	\begin{bmatrix}
		\dot{x} \\
		z \\
		e \\
		u \\
		y
	\end{bmatrix} &=
	\begin{bmatrix}[c|cc]
		A_H - m_{22}B_HC_H & -m_{21}B_H & B_H \\
		\cmidrule(lr){1-3}
		m_{12}C_H & m_{11} & 0 \\
		-m_{22}C_H & -m_{21} & 1 \\
		C_H & 0 & 0 \\
		m_{22}C_H & m_{21} & 0
	\end{bmatrix}
	\begin{bmatrix}
		x \\
		w \\
		r
	\end{bmatrix} \label{eq:aug} \\
	&=
	\begin{bmatrix}[c|cc]
		\mathcal{A} & \mathcal{B}_w & \mathcal{B}_r \\
		\cmidrule(lr){1-3}
		\mathcal{C}_z & \mathcal{D}_{zw} & \mathcal{D}_{zr} \\
		\mathcal{C}_e & \mathcal{D}_{ew} & \mathcal{D}_{er} \\
		\mathcal{C}_u & \mathcal{D}_{uw} & \mathcal{D}_{ur} \\
		\mathcal{C}_y & \mathcal{D}_{yw} & \mathcal{D}_{yr}
	\end{bmatrix}
	\begin{bmatrix}
		x \\
		w \\
		r
	\end{bmatrix} \label{eq:aug-ss}
\end{align}

With the channels of interest exposed and the system in a state-space form, one can express design goals as constraints on these channels. In \autoref{ch:Stability}, various stability measures and performance goals are discussed from which those that are ideal from an optimization perspective are selected. In \autoref{ch:Optimization}, the framework is introduced to allow the these goals to be applied to the augmented system in a way that allows the optimization problem to be efficiently solved.
%\include{impl}
%\include{discussion}
%%% The following is a directive for TeXShop to indicate the main file
%%!TEX root = diss.tex

\chapter{Conclusions}
\label{ch:Conclusions}

%    3. Notes
%    4. Footnotes

%    5. Bibliography
\begin{singlespace}
\raggedright
\bibliographystyle{abbrvnat}
\bibliography{biblio}
\end{singlespace}

\appendix
%    6. Appendices (including copies of all required UBC Research
%       Ethics Board's Certificates of Approval)
%\include{reb-coa}	% pdfpages is useful here
%% The following is a directive for TeXShop to indicate the main file
%%!TEX root = diss.tex

\chapter{Derivation of Matrix Inequalities with One Quadratic Term}
\label{sec:apd-a}

\section{Derivation of \glsentryshort{GKYP} Inequality with Arbitrary $\mathcal{D}$}
\label{sec:apd-a-1}

\begin{thm}
	Equation~\ref{eq:lmiinf} from \autoref{sec:opt-gkyp} is equivalent to the following:
	
	\begin{equation} \label{eq:lmiinf-equiv}
		\begin{bmatrix}
			-\Xi_{11} + aa^T & -\Xi_{12} + a & -\mathcal{C}_q^T - a\mathcal{D}_{qp}^T \\
			-\Xi_{12}^T + a^T & -\Xi_{22} + 1 & -\mathcal{D}_{qp}^T \\
			-\mathcal{C}_q - a^T\mathcal{D}_{qp} & -\mathcal{D}_{qp} & \gamma_\infty
		\end{bmatrix} \geq 0
	\end{equation}
	
	where (\ref{eq:lmiinf-equiv}) contains just one nonlinear term in variable $a$, and:
	
	\begin{equation*}
		\begin{bmatrix}
			\Xi_{11} & \Xi_{12} \\
			\Xi_{12}^T & \Xi_{22}
		\end{bmatrix} =
		\begin{bmatrix}
			I & a \\
			0 & 1
		\end{bmatrix}
		\begin{bmatrix}
			\mathcal{A} & \mathcal{B}_p \\
			I & 0
		\end{bmatrix}^T
		\left(\Phi \oplus P_\gamma + \Psi \oplus Q_\gamma\right)
		\begin{bmatrix}
			\mathcal{A} & \mathcal{B}_p \\
			I & 0
		\end{bmatrix} 
		\begin{bmatrix}
			I & a \\
			0 & 1
		\end{bmatrix}^T
	\end{equation*}
	\begin{align} \label{eq:pq}
		P_\gamma = \gamma_\infty^{-1}P && Q_\gamma = \gamma_\infty^{-1}Q.
	\end{align}
\end{thm}

\begin{proof}
	Starting from \autoref{eq:lmiinf}, the procedure mentioned in \autoref{sec:opt-ss} is followed to eliminate non-convex products in the first term of the \gls{LMI} \cite[Th. 1]{Li2014}:

	\begin{equation} \label{eq:a-1}
		-\begin{bmatrix}
			I & a \\
			0 & 1
		\end{bmatrix}
		\begin{bmatrix}
			\mathcal{A} & \mathcal{B}_p \\
			I & 0
		\end{bmatrix}^T
		f\left(\Phi, \Psi, P, Q\right)
		\begin{bmatrix}
			\mathcal{A} & \mathcal{B}_p \\
			I & 0
		\end{bmatrix} 
		\begin{bmatrix}
			I & a \\
			0 & 1
		\end{bmatrix}^T +
		\ldots 
%		-\begin{bmatrix}
%			I & \vec{a}_S \\
%			0 & 1
%		\end{bmatrix}
%		\begin{bmatrix}
%			\mathcal{C}_y & \mathcal{D}_{yx} \\
%			0 & I
%		\end{bmatrix}^T
%		\begin{bmatrix}
%			1 & 0 \\
%			0 & -\gamma_\infty
%		\end{bmatrix}
%		\begin{bmatrix}
%			\mathcal{C}_y & \mathcal{D}_{yx} \\
%			0 & I
%		\end{bmatrix}
%		\begin{bmatrix} 
%			I & \vec{a}_S \\
%			0 & 1
%		\end{bmatrix}^T 
		\geq 0.
	\end{equation}

	Let the notation $\Xi_{ij}$ be used for the linear part:
	
	\begin{equation} \label{eq:a-2}
		-\begin{bmatrix}
			\Xi_{11} & \Xi_{12} \\
			\Xi_{12}^T & \Xi_{22}
		\end{bmatrix} + \ldots
%		\\
%		-\begin{bmatrix}
%			I & \vec{a}_S \\
%			0 & 1
%		\end{bmatrix}
%		\begin{bmatrix}
%			\mathcal{C}_y & \mathcal{D}_{yx} \\
%			0 & I
%		\end{bmatrix}^T
%		\begin{bmatrix}
%			1 & 0 \\
%			0 & -\gamma_\infty
%		\end{bmatrix}
%		\begin{bmatrix}
%			\mathcal{C}_y & \mathcal{D}_{yx} \\
%			0 & I
%		\end{bmatrix}
%		\begin{bmatrix} 
%			I & \vec{a}_S \\
%			0 & 1
%		\end{bmatrix}^T
		\geq 0.
	\end{equation}

	\autoref{eq:a-2} may undergo a congruent transformation by $\gamma_\infty^{-\frac{1}{2}}I$ introducing a commutable factor of $\gamma_\infty^{-1}$ to every element. For the first summation term, the factor is absorbed into $Q$ and $P$ with the redefinition from \autoref{eq:pq} yielding:

	\begin{equation} \label{eq:a-3}
		-\begin{bmatrix}
			\Xi_{11} & \Xi_{12} \\
			\Xi_{12}^T & \Xi_{22}
		\end{bmatrix} - 
		\begin{bmatrix}
			I & a \\
			0 & 1
		\end{bmatrix}
		\begin{bmatrix}
			\mathcal{C}_q & \mathcal{D}_{qp} \\
			0 & I
		\end{bmatrix}^T
		\begin{bmatrix}
			\gamma_\infty^{-1} & 0 \\
			0 & -1
		\end{bmatrix}
		\begin{bmatrix}
			\mathcal{C}_y & \mathcal{D}_{qp} \\
			0 & I
		\end{bmatrix}
		\begin{bmatrix} 
			I & a \\
			0 & 1
		\end{bmatrix}^T
		\geq 0.
	\end{equation}

	Multiplying the inner factors in the second term of \autoref{eq:a-3} leads to:
	
	\begin{equation*}
		-\begin{bmatrix}
			\Xi_{11} & \Xi_{12} \\
			\Xi_{12}^T & \Xi_{22}
		\end{bmatrix}^T -
		\begin{bmatrix}
			I & a \\
			0 & 1
		\end{bmatrix}
		\begin{bmatrix}
			\gamma_\infty^{-1}\mathcal{C}_q^T\mathcal{C}_q & \gamma_\infty^{-1}\mathcal{C}_q^T\mathcal{D}_{qp} \\
			\gamma_\infty^{-1}\mathcal{D}_{qp}^T\mathcal{C}_q & \gamma_\infty^{-1}\mathcal{D}_{qp}^T\mathcal{D}_{qp} - 1
		\end{bmatrix}
		\begin{bmatrix} 
			I & a \\
			0 & 1
		\end{bmatrix}^T
		\geq 0
	\end{equation*}

	which can be expanded into:
	
	\begin{multline} \label{eq:a-5}
		-\begin{bmatrix}
			\Xi_{11} & \Xi_{12} \\
			\Xi_{12}^T & \Xi_{22}
		\end{bmatrix} -
		\begin{bmatrix}
			I & a \\
			0 & 1
		\end{bmatrix}
		\begin{bmatrix}
			I & a\mathcal{D}_{qp}^T \\
			0 & \mathcal{D}_{qp}^T
		\end{bmatrix}
		\begin{bmatrix}
			\mathcal{C}_q^T \\
			1
		\end{bmatrix}
		\gamma_\infty^{-1}
		\begin{bmatrix}
			\mathcal{C}_q^T \\
			1
		\end{bmatrix}^T
		\begin{bmatrix}
			I & a\mathcal{D}_{qp}^T \\
			0 & \mathcal{D}_{qp}^T
		\end{bmatrix}^T
		\begin{bmatrix} 
			I & a \\
			0 & 1
		\end{bmatrix}^T + \\
		+\begin{bmatrix}
			I & a \\
			0 & 1
		\end{bmatrix}
		\begin{bmatrix}
			0 & 0 \\
			0 & 1
		\end{bmatrix}
		\begin{bmatrix} 
			I & a \\
			0 & 1
		\end{bmatrix}^T
		\geq 0.
	\end{multline}
	
	The 3 outer factors multiplied with $\gamma_\infty^{-1}$ in the middle term of \autoref{eq:a-5} are then combined together and the last summation term is also multiplied through, resulting in the following:
	
	\begin{equation} \label{eq:a-6}
		-\begin{bmatrix}
			\Xi_{11} & \Xi_{12} \\
			\Xi_{12}^T & \Xi_{22}
		\end{bmatrix}
		-\begin{bmatrix}
			\mathcal{C}_q^T + a\mathcal{D}_{qp}^T \\
			\mathcal{D}_{qp}^T
		\end{bmatrix}
		\gamma_\infty^{-1}
		\begin{bmatrix}
			\mathcal{C}_q^T + a\mathcal{D}_{qp}^T \\
			\mathcal{D}_{qp}^T
		\end{bmatrix}^T +
		\begin{bmatrix}
			aa^T & a \\
			a^T & 1
		\end{bmatrix}
		\geq 0.
	\end{equation}
	
	The last summation term of \autoref{eq:a-6} is then added with the linear part $\Xi$. Because $\gamma_\infty > 0 \leftrightarrow \gamma_\infty^{-1} > 0$, a Schur complement taken around $\gamma_\infty$ allows \autoref{eq:a-6} to be written as the single matrix inequality shown in \autoref{eq:lmiinf-equiv}.

\end{proof}

\section{Derivation of $\mathcal{H}_2$ and $\ell_1$ Inequalities}
\label{sec:apd-a-2}

\begin{thm}
	\autoref{eq:lmi2-1} from \autoref{sec:opt-h2} and \autoref{eq:lmi1-1} from \autoref{sec:opt-l1} are equivalent to the following:
	
	\begin{equation} \label{eq:lmi2-lmi1-equiv}
		\begin{bmatrix}
			-\Xi_{11} + aa^T & -\Xi_{12} + a \\
			-\Xi_{12}^T + a^T & -\Xi_{22} + 1
		\end{bmatrix} \geq 0,
	\end{equation}
	
	where \autoref{eq:lmi2-lmi1-equiv} contains just one nonlinear term in variable $a$, and:
	\vspace{-0.25cm} % To correct for there being one line on a new page at the end of A.2.

	\begin{equation*}
		\begin{bmatrix}
			\Xi_{11} & \Xi_{12} \\
			\Xi_{12}^T & \Xi_{22}
		\end{bmatrix} =
		\begin{bmatrix}
			I & a \\
			0 & 1
		\end{bmatrix}
		\begin{bmatrix}
			\mathcal{A} & \mathcal{B}_p \\
			I & 0
		\end{bmatrix}^T
		f\left(\Phi, P_\gamma, \alpha\right)
		\begin{bmatrix}
			\mathcal{A} & \mathcal{B}_p \\
			I & 0
		\end{bmatrix} 
		\begin{bmatrix}
			I & a \\
			0 & 1
		\end{bmatrix}^T
	\end{equation*}

	\begin{gather} \label{eq:pq}
		f\left(\Phi, P_\gamma, \alpha\right) = 
		\begin{cases}
			\Phi \oplus P_\gamma & \textrm{for the $\mathcal{H}_2$ case} \\
			\left(\Phi + \begin{bmatrix} 0 & 0 \\ 0 & \alpha \end{bmatrix}\right) \oplus P_\gamma & \textrm{for the $\ell_1$ case}
		\end{cases} \\
		P_\gamma = \gamma_\infty^{-1}P.
	\end{gather}
\end{thm}

\begin{proof}
	Starting from either \autoref{eq:lmi2-1} or \autoref{eq:lmi1-1}, the procedure mentioned in \autoref{sec:opt-h2l1} is followed to eliminate non-convex products in the first term of the \gls{LMI} independent of $f\left(\Phi, P_\gamma, \alpha\right)$. The second summation term is the same in both \gls{LMI}s and simplifies to \autoref{eq:cv-3}. Combining these, the matrix inequality from \autoref{eq:lmi2-lmi1-equiv} is produced.
\end{proof}

\backmatter
%    7. Index
% See the makeindex package: the following page provides a quick overview
% <http://www.image.ufl.edu/help/latex/latex_indexes.shtml>


\end{document}
