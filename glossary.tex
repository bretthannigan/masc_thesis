%% The following is a directive for TeXShop to indicate the main file
%%!TEX root = diss.tex

% Acronyms
\newacronym{A/D}{A/D}{analog-to-digital}
\newacronym{D/A}{D/A}{digital-to-analog}
\newacronym{CT}{CT}{continuous-time}
\newacronym{DT}{DT}{discrete-time}

\newacronym{EEG}{EEG}{electroencephalography}
\newacronym{ECG}{ECG}{electrocardiography}
\newacronym{PPG}{PPG}{photoplethysmography}

\newacronym{OSR}{OSR}{oversampling ratio}
\newacronym{SQNR}{SQNR}{signal-to-quantization-noise ratio}

\newacronym{AAF}{AAF}{antialiasing filter}
\newacronym{LF}{LF}{loop filter}
\newacronym{DRF}{DRF}{digital reconstruction filter}
\newacronym{LF}{LF}{loop filter}

\newacronym[description={noise transfer function, equivalent to the sensitivity function}]{NTF}{NTF}{noise transfer function}
\newacronym[description={signal transfer function, equivalent to the complementary sensitivity function}]{STF}{STF}{signal transfer function}

\newacronym{CLANS}{CLANS}{closed-loop analysis of noise shaper}
\newacronym{LMI}{LMI}{linear matrix inequality}
\newacronym{GKYP}{GKYP}{generalized Kalman-Yakubovi\v{c}-Popov}
\newacronym{FIR}{FIR}{finite impulse response}
\newacronym{IIR}{IIR}{infinite impulse response}


% Symbols
\newglossaryentry{pq}{
	name = $P_Q$ ,
	description = In-band quantization noise power
}
\newglossaryentry{delta}{
	name = $\Delta$ ,
	description = Quantization step size
}
\newglossaryentry{order}{
	name = $n$,
	description = Filter order
}
\newglossaryentry{r}{
	name = $r$,
	description = Analog reference input signal
}
\newglossaryentry{e}{
	name = $e$,
	description = Feedback error signal
}
\newglossaryentry{u}{
	name = $u$,
	description = Quantizer input signal
}
\newglossaryentry{y}{
	name = $y$,
	description = Digital bitstream output signal
}
\newglossaryentry{S}{
	name = $S(\lambda)$,
	description = Sensitivity function
}
\newglossaryentry{T}{
	name = $T(\lambda)$,
	description = Complementary sensitivity function
}
\newglossaryentry{sorz}{
	name = $\lambda$,
	description = Placeholder for the continuous-time Laplace variable $s$ or discrete-time $z$-transformation variable $z$
}

\renewcommand{\glsnamefont}[1]{{\scshape \MakeTextLowercase{#1}}}
\printglossary[title=List of Symbols]
\printglossary[type=\acronymtype,title=Glossary]

% You can also use \newacro{}{} to only define acronyms
% but without explictly creating a glossary
% 
% \newacro{ANOVA}[ANOVA]{Analysis of Variance\acroextra{, a set of
%   statistical techniques to identify sources of variability between groups.}}
% \newacro{API}[API]{application programming interface}
% \newacro{GOMS}[GOMS]{Goals, Operators, Methods, and Selection\acroextra{,
%   a framework for usability analysis.}}
% \newacro{TLX}[TLX]{Task Load Index\acroextra{, an instrument for gauging
%   the subjective mental workload experienced by a human in performing
%   a task.}}
% \newacro{UI}[UI]{user interface}
% \newacro{UML}[UML]{Unified Modelling Language}
% \newacro{W3C}[W3C]{World Wide Web Consortium}
% \newacro{XML}[XML]{Extensible Markup Language}
